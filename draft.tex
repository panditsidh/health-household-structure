\documentclass[12pt]{article}
\usepackage{graphicx} %package to manage images
\graphicspath{ {./figures/} }
\usepackage{caption}
\usepackage[font=scriptsize]{subcaption}
\captionsetup[figure]{labelsep=none}
\captionsetup[table]{labelsep=none}
\usepackage{bbm}
\usepackage{amsmath}
\usepackage{import}
\usepackage{array}
\usepackage{booktabs}     % no extra vertical space between paragraphs
\usepackage{indentfirst}
\usepackage{afterpage}
\usepackage{floatrow}
\usepackage{pdflscape}
\usepackage{soul}
\usepackage{float}
\usepackage{adjustbox}
\usepackage{longtable}
\usepackage{caption}
\usepackage{setspace}
\usepackage{afterpage}
\usepackage[margin=1in]{geometry}
\usepackage[round]{natbib}
\usepackage{hyperref}
\usepackage{titlesec}
\usepackage{threeparttable}
\usepackage{setspace}
\usepackage{float}
\floatstyle{plaintop}
\restylefloat{table}
\usepackage{rotating}
\usepackage{authblk}
\usepackage{makecell}


% Optional styling for nicer author/affil typography
\renewcommand\Authfont{\normalsize\bfseries}
\renewcommand\Affilfont{\small\normalfont}
\setlength{\affilsep}{0.3em} % space between author and affiliation

\setlength{\parindent}{0pt}   % standard indent (~0.25 inch)
\setlength{\parskip}{1em}       % no extra vertical space between paragraphs

\titleformat{\section}
  {\normalfont\bfseries\fontsize{14}{16}\selectfont}  % Bold 14pt
  {}{0pt}{}  % No numbering

\titleformat{\subsection}
  {\normalfont\bfseries\fontsize{12}{14}\selectfont}  % Bold 12pt
  {}{0pt}{}  % No numbering

\titleformat{\subsubsection}
  {\normalfont\itshape\fontsize{12}{14}\selectfont}  % Italic 12pt
  {}{0pt}{}  % No numbering
  
\begin{document}
\title{Pregnant women's health and health decision making in Indian households}
\date{\today}
\maketitle

\section{Introduction}

Health during pregnancy matters for both women and their children.  The ability to make decisions about your own healthcare may have positive consequences for health and is a good in itself.  

Both theory (modernization theory) and evidence (Allendorf, 2012) suggest a trend towards nuclear households in India.  Scholars have long associated nuclear households with more autonomy for women (cite Jeffreys, others).  

This paper investigates trends in household structure among pregnant women in India in 3 rounds of the NFHS.  Contrary to the modernization literature, we find an increasing proportion of pregnant women live in patrilocal joint households, and that a shrinking proportion live in nuclear households.  The proportion living in natal households remained stable between 2005 and 2020.  

The primary contribution of this paper is to investigate the consequences of this demographic shift for pregnant women's health and health care decision making.  We further point researchers interested in understanding the causes of increasing co-residence of pregnant women with their in-laws towards some possible candidate explanations.

The increase in joint patrilocal households seems not to have had negative consequences for women's decision making about their own health care.  Pregnant women's autonomy as measured by their say in healthcare decisions has increased significantly in all household types, and gaps have narrowed. Women in joint patrilocal are now only x\% pp less likely to report say in their own decisions than women in nuclear households, compared to X\% in 2005.

Consistent with the prior literature which studied trends in the 1990s and early 2000s, we find some health advantages of joint patrilocal living: women in joint patrilocal households have greater dairy consumption and are less likely to experience physical violence by their husbands.  (BMI?)  Another difference, which we investigate in further detail, is that recently delivered women living in joint patrilocal households are more likely to give birth in private facilities than those living in nuclear households.  This is consistent with them living in wealthier households.

Despite families' likely assumptions that more costly care is better care, joint families' choice of private providers may actually be harmful for pregnant women and children in parts of the country where private providers have worse outcomes than public providers.  (summary of an analysis about neonatal or early neonatal death regressed on household type for different regions, control for facility type, see if it disappears.  Try the same for csection in south...)

Given that the demographic trends that appear to be contributing to an increase in joint family living for pregnant women seem likely to continue (cite the other paper), and that nuclear families will get richer and be more likely to be able to afford private care, a policy recommendation is to positively norm pregnant women being capable of making their own decisions, while educating everyone in pregnant women's households about what good delivery care looks like (qualified provider, less intervention).


\section{Background}

\subsection{The significance of women's healthcare decision making}



\subsection{Women's health, autonomy, and household structure: Are there causal relationships?}

\subsection{The changing experience of pregnancy}

\subsection{Trends in the use of healthcare at birth}

\section{Data and Methods}

Describe the NFHS data.

\section{Results}

\begin{table}[H]
    \centering
    \setlength{\tabcolsep}{4pt} % shrink column padding
    \footnotesize % shrink text
    \caption{: Household structure 3+ mopreg women are observed in by subgroup}
    \label{tab:sumstat}
    \begin{adjustbox}{width=\textwidth}
        \begin{tabular}{lcccc}
\toprule
Group & NFHS-2 & NFHS-3 & NFHS-4 & NFHS-5 \\\\
\midrule
\textbf{Nuclear Households}&&&&\\
India&29.4 (28.0, 30.9)&32.9 (31.1, 34.7)&26.1 (25.4, 26.9)&24.6 (23.9, 25.3)\\
Rural&29.6 (26.7, 32.6)&33.7 (30.8, 36.6)&28.1 (26.3, 29.9)&26.5 (24.8, 28.3)\\
Urban&29.4 (27.7, 31.0)&32.6 (30.5, 34.8)&25.4 (24.6, 26.1)&23.9 (23.1, 24.7)\\
Forward Caste&36.5 (31.5, 41.5)&40.2 (35.1, 45.2)&30.5 (28.6, 32.5)&29.7 (27.0, 32.4)\\
OBC&32.0 (28.7, 35.3)&38.8 (35.1, 42.6)&28.4 (26.7, 30.0)&25.9 (24.4, 27.4)\\
Dalit&26.8 (24.2, 29.4)&29.9 (27.0, 32.9)&23.7 (22.6, 24.8)&20.9 (19.8, 22.1)\\
Adivasi&25.0 (22.2, 27.7)&24.0 (20.7, 27.3)&20.2 (18.2, 22.2)&19.9 (17.8, 21.9)\\
Muslim&33.2 (29.3, 37.1)&38.2 (33.2, 43.2)&31.7 (29.8, 33.7)&32.0 (30.0, 34.1)\\
Sikh, Jain, Christian&25.8 (17.2, 34.4)&20.2 (11.9, 28.6)&18.5 (13.3, 23.6)&17.3 (11.1, 23.6)\\
.&&&&\\
\textbf{Sasural Households}&&&&\\
India&54.3 (52.8, 55.8)&51.7 (49.9, 53.5)&58.5 (57.6, 59.3)&59.9 (59.1, 60.8)\\
Rural&53.2 (50.1, 56.3)&51.8 (48.7, 54.9)&56.0 (53.9, 58.1)&57.8 (56.0, 59.7)\\
Urban&54.6 (52.9, 56.4)&51.7 (49.6, 53.9)&59.5 (58.6, 60.3)&60.7 (59.7, 61.6)\\
Forward Caste&51.2 (46.3, 56.1)&46.3 (41.1, 51.5)&55.9 (53.7, 58.1)&58.3 (55.6, 60.9)\\
OBC&54.4 (51.0, 57.8)&48.5 (44.8, 52.2)&57.1 (55.3, 58.9)&58.6 (56.9, 60.2)\\
Dalit&55.1 (52.3, 58.0)&52.3 (48.9, 55.6)&60.3 (59.1, 61.6)&62.5 (61.2, 63.8)\\
Adivasi&57.4 (54.3, 60.4)&60.7 (57.0, 64.4)&64.8 (62.6, 67.1)&64.6 (61.9, 67.2)\\
Muslim&49.1 (45.0, 53.2)&47.0 (42.5, 51.5)&51.6 (49.4, 53.8)&53.1 (51.0, 55.3)\\
Sikh, Jain, Christian&62.8 (52.2, 73.4)&64.6 (54.0, 75.1)&65.4 (59.0, 71.8)&62.4 (54.6, 70.1)\\
.&&&&\\
\textbf{Natal Households}&&&&\\
India&16.3 (15.1, 17.5)&15.4 (14.1, 16.6)&15.4 (14.8, 16.0)&15.5 (14.8, 16.1)\\
Rural&17.2 (14.7, 19.6)&14.5 (12.3, 16.7)&15.9 (14.5, 17.3)&15.6 (14.2, 17.0)\\
Urban&16.0 (14.7, 17.4)&15.7 (14.2, 17.1)&15.2 (14.5, 15.8)&15.4 (14.7, 16.1)\\
Forward Caste&12.3 (9.4, 15.1)&13.5 (10.1, 17.0)&13.5 (12.0, 15.1)&12.0 (10.5, 13.5)\\
OBC&13.6 (11.1, 16.1)&12.7 (10.3, 15.0)&14.5 (13.3, 15.8)&15.5 (14.3, 16.7)\\
Dalit&18.0 (15.6, 20.5)&17.8 (15.4, 20.1)&16.0 (15.0, 16.9)&16.5 (15.5, 17.6)\\
Adivasi&17.7 (15.3, 20.1)&15.2 (12.5, 18.0)&14.9 (13.2, 16.6)&15.5 (13.1, 18.0)\\
Muslim&17.7 (14.5, 20.9)&14.7 (11.5, 18.0)&16.7 (15.0, 18.4)&14.8 (13.2, 16.4)\\
Sikh, Jain, Christian&11.4 (5.2, 17.6)&15.2 (8.1, 22.4)&16.2 (11.4, 20.9)&20.3 (13.4, 27.2)\\
.&&&&\\
&&&&\\
\bottomrule
\end{tabular}

    \end{adjustbox}
\end{table}

\section{Discussion}

\newpage

\section{References}


\newpage



\section{Tables and figures}

\begin{figure}[H]
    \centering
    \includegraphics[width=\textwidth]{figures/hhstruc pregnant.png}
\end{figure}

\begin{figure}[H]
    \centering
    \includegraphics[width=\textwidth]{figures/No say in own healthcare_pregnant.png}

    \parbox{1\linewidth}{\footnotesize Notes: NFHS-2 asks "who decides on obtaining health care". NFHS-3 asks "Final say on own health care". NFHS-4/5 asks "person who usually decides on respondent's healthcare"}
\end{figure}


\begin{figure}[H]
    \centering
    \includegraphics[width=\textwidth]{figures/No say in large purchases_pregnant.png}
    \parbox{1\linewidth}{\footnotesize Notes: NFHS-2 asks "Who decides to purchase jewelry". NFHS-3 asks "Final say on making large household purchases". NFHS-4/5 asks "Person who usually decides on large household purchases"}
\end{figure}


\begin{figure}[H]
    \centering
    \includegraphics[width=\textwidth]{figures/Experienced physical domestic violence.png}
    \parbox{1\linewidth}{\footnotesize Notes: NFHS-2 asks "has been beaten since age 15, and husband has beaten respondent". NFHS-3/4/5 asks "spouse ever pushed/slapped/punched etc.}
\end{figure}

\begin{figure}[H]
    \centering
    \includegraphics[width=\textwidth]{figures/Consumes dairy daily_pregnant.png}
\end{figure}



\begin{figure}[H]
    \centering
    \includegraphics[width=\textwidth]{figures/hhstruc_bmi_preg.png}
\end{figure}


\begin{figure}[H]
    \centering
    \includegraphics[width=\textwidth]{figures/Wealth index z score.png}
\end{figure}

\begin{figure}[H]
    \centering
    \includegraphics[width=\textwidth]{figures/Delivered by Csection last 3-12 mo._nonpregnant.png}
\end{figure}

\begin{figure}[H]
    \centering
    \includegraphics[width=\textwidth]{figures/Last birth occurred at home within 3-12 mo._nonpregnant.png}
\end{figure}

\begin{figure}[H]
    \centering
    \includegraphics[width=\textwidth]{figures/region facility round2.png}
\end{figure}

\begin{figure}[H]
    \centering
    \includegraphics[width=\textwidth]{figures/region facility round3.png}
\end{figure}

\begin{figure}[H]
    \centering
    \includegraphics[width=\textwidth]{figures/region facility round4.png}
\end{figure}

\begin{figure}[H]
    \centering
    \includegraphics[width=\textwidth]{figures/region facility round5.png}
\end{figure}




\section{Appendix}



\begin{figure}[H]
    \centering
    \includegraphics[width=\textwidth]{figures/hhstruc non pregnant.png}
\end{figure}

\begin{table}[H]
    \centering
    \setlength{\tabcolsep}{4pt} % shrink column padding
    \footnotesize % shrink text
    \caption{: Household structure 3+ mopreg women are observed in by subgroup}
    \label{tab:sumstat}
    \begin{adjustbox}{width=\textwidth}
        \begin{tabular}{lcccc}
\toprule
Group & NFHS-2 & NFHS-3 & NFHS-4 & NFHS-5 \\\\
\midrule
\textbf{Nuclear Households}&&&&\\
India&29.4 (28.0, 30.9)&32.9 (31.1, 34.7)&26.1 (25.4, 26.9)&24.6 (23.9, 25.3)\\
Rural&29.6 (26.7, 32.6)&33.7 (30.8, 36.6)&28.1 (26.3, 29.9)&26.5 (24.8, 28.3)\\
Urban&29.4 (27.7, 31.0)&32.6 (30.5, 34.8)&25.4 (24.6, 26.1)&23.9 (23.1, 24.7)\\
Forward Caste&36.5 (31.5, 41.5)&40.2 (35.1, 45.2)&30.5 (28.6, 32.5)&29.7 (27.0, 32.4)\\
OBC&32.0 (28.7, 35.3)&38.8 (35.1, 42.6)&28.4 (26.7, 30.0)&25.9 (24.4, 27.4)\\
Dalit&26.8 (24.2, 29.4)&29.9 (27.0, 32.9)&23.7 (22.6, 24.8)&20.9 (19.8, 22.1)\\
Adivasi&25.0 (22.2, 27.7)&24.0 (20.7, 27.3)&20.2 (18.2, 22.2)&19.9 (17.8, 21.9)\\
Muslim&33.2 (29.3, 37.1)&38.2 (33.2, 43.2)&31.7 (29.8, 33.7)&32.0 (30.0, 34.1)\\
Sikh, Jain, Christian&25.8 (17.2, 34.4)&20.2 (11.9, 28.6)&18.5 (13.3, 23.6)&17.3 (11.1, 23.6)\\
.&&&&\\
\textbf{Sasural Households}&&&&\\
India&54.3 (52.8, 55.8)&51.7 (49.9, 53.5)&58.5 (57.6, 59.3)&59.9 (59.1, 60.8)\\
Rural&53.2 (50.1, 56.3)&51.8 (48.7, 54.9)&56.0 (53.9, 58.1)&57.8 (56.0, 59.7)\\
Urban&54.6 (52.9, 56.4)&51.7 (49.6, 53.9)&59.5 (58.6, 60.3)&60.7 (59.7, 61.6)\\
Forward Caste&51.2 (46.3, 56.1)&46.3 (41.1, 51.5)&55.9 (53.7, 58.1)&58.3 (55.6, 60.9)\\
OBC&54.4 (51.0, 57.8)&48.5 (44.8, 52.2)&57.1 (55.3, 58.9)&58.6 (56.9, 60.2)\\
Dalit&55.1 (52.3, 58.0)&52.3 (48.9, 55.6)&60.3 (59.1, 61.6)&62.5 (61.2, 63.8)\\
Adivasi&57.4 (54.3, 60.4)&60.7 (57.0, 64.4)&64.8 (62.6, 67.1)&64.6 (61.9, 67.2)\\
Muslim&49.1 (45.0, 53.2)&47.0 (42.5, 51.5)&51.6 (49.4, 53.8)&53.1 (51.0, 55.3)\\
Sikh, Jain, Christian&62.8 (52.2, 73.4)&64.6 (54.0, 75.1)&65.4 (59.0, 71.8)&62.4 (54.6, 70.1)\\
.&&&&\\
\textbf{Natal Households}&&&&\\
India&16.3 (15.1, 17.5)&15.4 (14.1, 16.6)&15.4 (14.8, 16.0)&15.5 (14.8, 16.1)\\
Rural&17.2 (14.7, 19.6)&14.5 (12.3, 16.7)&15.9 (14.5, 17.3)&15.6 (14.2, 17.0)\\
Urban&16.0 (14.7, 17.4)&15.7 (14.2, 17.1)&15.2 (14.5, 15.8)&15.4 (14.7, 16.1)\\
Forward Caste&12.3 (9.4, 15.1)&13.5 (10.1, 17.0)&13.5 (12.0, 15.1)&12.0 (10.5, 13.5)\\
OBC&13.6 (11.1, 16.1)&12.7 (10.3, 15.0)&14.5 (13.3, 15.8)&15.5 (14.3, 16.7)\\
Dalit&18.0 (15.6, 20.5)&17.8 (15.4, 20.1)&16.0 (15.0, 16.9)&16.5 (15.5, 17.6)\\
Adivasi&17.7 (15.3, 20.1)&15.2 (12.5, 18.0)&14.9 (13.2, 16.6)&15.5 (13.1, 18.0)\\
Muslim&17.7 (14.5, 20.9)&14.7 (11.5, 18.0)&16.7 (15.0, 18.4)&14.8 (13.2, 16.4)\\
Sikh, Jain, Christian&11.4 (5.2, 17.6)&15.2 (8.1, 22.4)&16.2 (11.4, 20.9)&20.3 (13.4, 27.2)\\
.&&&&\\
&&&&\\
\bottomrule
\end{tabular}

    \end{adjustbox}
\end{table}

\begin{table}[!htbp]
    \centering
    \begin{threeparttable}[t]
        \caption{:  Proportion of pregnant women who report having no say in own healthcare}
        \label{tab:sumstats}

        \scriptsize
        \setlength{\tabcolsep}{2pt}
        \renewcommand{\arraystretch}{1.2}

        \begin{tabular}{l*{3}{>{\centering\arraybackslash}p{1.4cm}}}
\toprule
 & Nuclear & Patrilocal & Natal \\\\
\midrule
\textbf{NFHS-2 (1998–99)}&&&\\
\hspace*{2em}Adivasi&.57&.63&.58\\
\hspace*{2em}Dalit&.55&.56&.56\\
\hspace*{2em}OBC&.53&.57&.56\\
\hspace*{2em}Forward caste&.56&.56&.62\\
\hspace*{2em}Muslim&.55&.58&.60\\
&&&\\
\textbf{NFHS-3 (2005–06)}&&&\\
\hspace*{2em}Adivasi&.42&.58&.48\\
\hspace*{2em}Dalit&.33&.50&.51\\
\hspace*{2em}OBC&.35&.57&.53\\
\hspace*{2em}Forward caste&.31&.50&.49\\
\hspace*{2em}Muslim&.38&.55&.51\\
&&&\\
\textbf{NFHS-4 (2015–16)}&&&\\
\hspace*{2em}Adivasi&.25&.28&.24\\
\hspace*{2em}Dalit&.28&.33&.21\\
\hspace*{2em}OBC&.27&.35&.18\\
\hspace*{2em}Forward caste&.26&.33&.22\\
\hspace*{2em}Muslim&.23&.31&.18\\
&&&\\
\textbf{NFHS-5 (2019–21)}&&&\\
\hspace*{2em}Adivasi&.17&.21&.11\\
\hspace*{2em}Dalit&.20&.23&.20\\
\hspace*{2em}OBC&.20&.28&.17\\
\hspace*{2em}Forward caste&.16&.23&.17\\
\hspace*{2em}Muslim&.25&.31&.24\\
\bottomrule
\end{tabular}

                
        \begin{tablenotes}
            \item NFHS-2: who decides on obtaining healthcare
            \item NFHS-3: final say on own healthcare
            \item NFHS-4/5: person who usually decides on respondent's healthcare
        \end{tablenotes}
    \end{threeparttable}
\end{table}


\begin{figure}[H]
    \centering
    \includegraphics[width=\textwidth]{figures/No say in visiting natal family_pregnant.png}
    \parbox{1\linewidth}{\footnotesize Notes: NFHS-2 asks "permission needed to visit relatives or friends. NFHS-3 asks "final say on visit to family or relatives". NFHS-4/5 asks "person who usually decides on visits to family or relatives}
\end{figure}


\begin{figure}[H]
    \centering
    \includegraphics[width=\textwidth]{figures/Consumes meat,egg,fish daily_pregnant.png}
\end{figure}


\begin{figure}[H]
    \centering
    \includegraphics[width=\textwidth]{figures/Any anemia (DHS cutoff for pregnancy)_pregnant.png}
\end{figure}


\begin{figure}[H]
    \centering
    \includegraphics[width=\textwidth]{figures/Experienced sexual domestic violence_pregnant.png}
\end{figure}

\end{document}
