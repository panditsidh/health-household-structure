\documentclass[12pt]{article}
\usepackage{graphicx} %package to manage images
\graphicspath{ {./figures/} }
\usepackage{caption}
\usepackage[font=scriptsize]{subcaption}
\captionsetup[figure]{labelsep=none}
\captionsetup[table]{labelsep=none}
\usepackage{bbm}
\usepackage{amsmath}
\usepackage{import}
\usepackage{array}
\usepackage{booktabs}     % no extra vertical space between paragraphs
\usepackage{indentfirst}
\usepackage{afterpage}
\usepackage{floatrow}
\usepackage{pdflscape}
\usepackage{soul}
\usepackage{float}
\usepackage{adjustbox}
\usepackage{longtable}
\usepackage{caption}
\usepackage{setspace}
\usepackage{afterpage}
\usepackage[margin=1in]{geometry}
\usepackage[round]{natbib}
\setcitestyle{authoryear}
\usepackage{hyperref}
\usepackage{titlesec}
\usepackage{threeparttable}
\usepackage{setspace}
\usepackage{float}
\floatstyle{plaintop}
\restylefloat{table}
\usepackage{rotating}
\usepackage{authblk}
\usepackage{makecell}


% Optional styling for nicer author/affil typography
\renewcommand\Authfont{\normalsize\bfseries}
\renewcommand\Affilfont{\small\normalfont}
\setlength{\affilsep}{0.3em} % space between author and affiliation

\setlength{\parindent}{0pt}   % standard indent (~0.25 inch)
\setlength{\parskip}{1em}       % no extra vertical space between paragraphs

\titleformat{\section}
  {\normalfont\bfseries\fontsize{14}{16}\selectfont}  % Bold 14pt
  {}{0pt}{}  % No numbering

\titleformat{\subsection}
  {\normalfont\bfseries\fontsize{12}{14}\selectfont}  % Bold 12pt
  {}{0pt}{}  % No numbering

\titleformat{\subsubsection}
  {\normalfont\itshape\fontsize{12}{14}\selectfont}  % Italic 12pt
  {}{0pt}{}  % No numbering
  
\begin{document}
\title{Consequence of increasing joint family residence on pregnant women's autonomy and healthcare in India: A Research Note}
\date{\today}
\maketitle

\section{Introduction}

Health during pregnancy matters for both women and their children.  The ability to make decisions about your own healthcare may have positive consequences for health and is a good in itself.  

Both theory (modernization theory) and evidence (\cite{allendorf2013going}) suggest a trend towards nuclear households in India.  Scholars have long associated nuclear households with more autonomy for women (\cite{jacobson1992women};\cite{jeffery1989labour};\cite{madan1989family}).  

Research questions: What is the relationship between household structure and pregnant women's autonomy (say in decisions about own health, say in visits to family and friends, and getting permission is a problem)?  What is the relationship between household structure and pregnant women's health outcomes (Nutrition: dairy consumption, meat consumption, BMI, underweight, anemia; Violence: sexual, physical;  Pregnancy and birth care and outcomes: ANC visits, home birth, c-section, NNM, (rural north only): private hospital)?  

Short description of findings: Pregnant women in joint households have less say.  Pregnant women in joint households have better pregnancy health outcomes, which are mostly accounted for by wealth differences.

The increase in joint household living is not impacting autonomy much because there is a countervailing improvement in autonomy for everyone, and more so for women in joint households -- if joint HH had stayed the same, only 4\% more women would have say in health care and only x\% more women would have say in visits to family and friends.

The increase in joint household living is...[do a similar decomposition with our favorite outcomes]



**

This paper investigates trends in household structure among pregnant women in India in 3 rounds of the NFHS.  Contrary to the modernization literature, we find an increasing proportion of pregnant women live in patrilocal joint households, and that a shrinking proportion live in nuclear households.  The proportion living in natal households remained stable between 2005 and 2020.  

The primary contribution of this paper is to investigate the consequences of this demographic shift for pregnant women's health and health care decision making.  We further point researchers interested in understanding the causes of increasing co-residence of pregnant women with their in-laws towards some possible candidate explanations.

The increase in joint patrilocal households seems not to have had negative consequences for women's decision making about their own health care.  Pregnant women's autonomy as measured by their say in healthcare decisions has increased significantly in all household types, and gaps have narrowed. Women in joint patrilocal are now only x\% pp less likely to report say in their own decisions than women in nuclear households, compared to X\% in 2005.

Consistent with the prior literature which studied trends in the 1990s and early 2000s, we find some health advantages of joint patrilocal living: women in joint patrilocal households have greater dairy consumption and are less likely to experience physical violence by their husbands.  (BMI?)  Another difference, which we investigate in further detail, is that recently delivered women living in joint patrilocal households are more likely to give birth in private facilities than those living in nuclear households.  This is consistent with them living in wealthier households.

Despite families' likely assumptions that more costly care is better care, joint families' choice of private providers may actually be harmful for pregnant women and children in parts of the country where private providers have worse outcomes than public providers.  (summary of an analysis about neonatal or early neonatal death regressed on household type for different regions, control for facility type, see if it disappears.  Try the same for csection in south...)

Given that the demographic trends that appear to be contributing to an increase in joint family living for pregnant women seem likely to continue (cite the other paper), and that nuclear families will get richer and be more likely to be able to afford private care, a policy recommendation is to positively norm pregnant women being capable of making their own decisions, while educating everyone in pregnant women's households about what good delivery care looks like (qualified provider, less intervention).
Health during pregnancy matters for both women and their children.  The ability to make decisions about your own healthcare may have positive consequences for health and is a good in itself.  

Both theory (modernization theory) and evidence (Allendorf, 2012) suggest a trend towards nuclear households in India.  Scholars have long associated nuclear households with more autonomy for women (cite Jeffreys, others).  

This paper investigates trends in household structure among pregnant women in India in 3 rounds of the NFHS.  Contrary to the modernization literature, we find an increasing proportion of pregnant women live in patrilocal joint households, and that a shrinking proportion live in nuclear households.  The proportion living in natal households remained stable between 2005 and 2020.  

The primary contribution of this paper is to investigate the consequences of this demographic shift for pregnant women's health and health care decision making.  We further point researchers interested in understanding the causes of increasing co-residence of pregnant women with their in-laws towards some possible candidate explanations.

The increase in joint patrilocal households seems not to have had negative consequences for women's decision making about their own health care.  Pregnant women's autonomy as measured by their say in healthcare decisions has increased significantly in all household types, and gaps have narrowed. Women in joint patrilocal are now only x\% pp less likely to report say in their own decisions than women in nuclear households, compared to X\% in 2005.

Consistent with the prior literature which studied trends in the 1990s and early 2000s, we find some health advantages of joint patrilocal living: women in joint patrilocal households have greater dairy consumption and are less likely to experience physical violence by their husbands.  (BMI?)  Another difference, which we investigate in further detail, is that recently delivered women living in joint patrilocal households are more likely to give birth in private facilities than those living in nuclear households.  This is consistent with them living in wealthier households.

Despite families' likely assumptions that more costly care is better care, joint families' choice of private providers may actually be harmful for pregnant women and children in parts of the country where private providers have worse outcomes than public providers.  (summary of an analysis about neonatal or early neonatal death regressed on household type for different regions, control for facility type, see if it disappears.  Try the same for csection in south...)

Given that the demographic trends that appear to be contributing to an increase in joint family living for pregnant women seem likely to continue (cite the other paper), and that nuclear families will get richer and be more likely to be able to afford private care, a policy recommendation is to positively norm pregnant women being capable of making their own decisions, while educating everyone in pregnant women's households about what good delivery care looks like (qualified provider, less intervention).


\section{Background}

\subsection{The significance of women's healthcare decision making}
Women's autonomy or decision-making power within their households has a significant effect on their healthcare, especially when it comes to neonatal health and maternal healthcare. \cite{coffey2022mothers} show that children of lower-ranking mothers are more likely to die in early life. In joint patrilocal households in rural India women who are married to the younger brothers are assigned lower rank than women married to the older brother in the same household. As numerous healthcare policies are rooted in women empowerment evidence from this paper indicates the role of social ranking in empowerment. Further, \cite{coffey2021neonatal} emphasizes the role of decision making power in case of neonatal mortality through maternal under nutrition which is concentrated among younger women who are early in their careers of childbearing. The findings also indicate the relevance of social status of women, household structure, and demographic relations in health outcomes in the developing world. 


\subsection{Women's health, autonomy, and household structure: Are there causal relationships?}
There is an increasing trend in decision making power among women in the last three rounds of NFHS. But at the same time we did observe an increase in young women's residence in the patrilocal extended households unlike \cite{allendorf2013going} where she observed an increase in young married women's residence in nuclear households. But their work emphasized that contrary to literature young women living in the nuclear households do not have better health than those living in patrilocal extended families. Further, their work showed that women living in patrilocal residences had an overall advantage Although this paper also notes that women living in nuclear families are more likely to consume meat, fish and eggs on a weekly basis but there is an overall advantage for women living in patrilocal extended families. Although \cite{hou2013effect} work doesn't show a direct association between household structure and women's health and autonomy but their work indicates that the presence of influential males' have an opposite effect in uptake of maternal health service uptake. 

\subsection{The changing experience of pregnancy}
Trends in decision making power indicate increase in involvement of the women in different forms of decision making in the household including their own healthcare. Studies by \cite{ghose2017women} using Bangladesh DHS indicate that women who make decisions along with their spouses are more likely to use more number of maternal health services. Thus, indicating an increase as well as advantage in husband's involvement in reproductive care. 

\subsection{Trends in the use of healthcare at birth}
There has been a significant change in use of healthcare at birth in India. Though there has been an increase in uptake of maternal healthcare services, there has also been a doubling in cesarean deliveries. This growth is closely related to the simultaneous decline in the share of home deliveries without medical supervision and an increase in institutional deliveries spearheaded by contemporary cash transfer policies encouraging women to use prenatal and postnatal care in modern health centers. Along the same time the major rise in cesarean deliveries occurred in private clinics where it rose from 27.5 percent to 40.8 percent as opposed to government facilities where it remained 11 percent in 2015-16(\cite{guilmoto2019trends}. Other studies such as \cite{singh2021maternal}'s study using NFHS data shows that there is still a persisting difference in the full use of ANC (antenatal care) and SBA (skilled birth attendant) especially among adolescent mothers. They also observed that the likelihood of using these services also decreases with increase in birth order. In the context of utilizing maternal healthcare services \cite{pathak2010economic} observed economic status as one of the parameters of utilizing health services. Even \cite{allendorf2013going} observes the same as in her paper she identifies economic status as a mediator for the patrilocal residence advantage where women in patrilocal extended households are more likely to use antenatal care and delivery assistance as well as consume milk and curd weekly and are also less likely to report physical violence


\section{Data and Methods}

This paper uses data from the National Family Health Survey (NFHS) collected in 2005-2006, 2015-16 and 2019-21. It is a cross-sectional, large-scale, nationally representative survey of households and women of reproductive age. I have restricted the sample to only pregnant women as here the focus is on pregnant women's access and experience in healthcare. Further, I have constructed the household structures using information from the household roster which provides data on who else resides in the same household as the respondent. The analytical sample consists of 5911 pregnant women interviewed in 2005-2006; 32,428 pregnant women interviewed in 2015-16, and 28,408 pregnant women interviewed in 2019-21. 

\section{Results}




\section{Discussion}

\newpage




\newpage


\section{Tables and figures}

\begin{figure}[H]
    \centering
    \includegraphics[width=\textwidth]
    {figures/figure 1 change in hhstructure.png}
    \caption{: 1 panel figure showing change showing household structure}
    \label{fig:ihds_eat_last}
    \parbox{1\linewidth}{\footnotesize figure notes go here}
\end{figure}


\begin{landscape}
    

\begin{figure}[H]
    \centering
    \includegraphics[width=\textwidth]
    {figures/figure 2 four panel.pdf}
    \caption{: 4 panel figure showing changes over time in say in own healthcare, say in visits to family, friends, facility birth, and 4+ ANC visits}
    \label{fig:ihds_eat_last}
    \parbox{1\linewidth}{\footnotesize figure notes go here}
\end{figure}

\end{landscape}

\begin{landscape}

\begin{table}[!htbp]
    \centering
    \begin{threeparttable}[t]
        \caption{:  Summary statistics for variables used in mediation analysis}
        \label{tab:sumstats}

        \scriptsize
        \setlength{\tabcolsep}{6pt}
        \renewcommand{\arraystretch}{1}

        \begin{tabular}{lcccccc}
\toprule
 & \multicolumn{2}{c}{NFHS-3} & \multicolumn{2}{c}{NFHS-4} & \multicolumn{2}{c}{NFHS-5} \\\\
\cmidrule(lr){2-3} \cmidrule(lr){4-5} \cmidrule(lr){6-7}
 & Joint & Nuclear & Joint & Nuclear & Joint & Nuclear \\\\
\midrule
\textbf{Currently pregnant women}&&&&&&\\
&&&&&&\\
\textbf{Autonomy measures}&&&&&&\\
\hspace*{2em}Say in own healthcare&0.32&0.51&0.26&0.32&0.18&0.25\\
\hspace*{2em}Say in visits to family/friends&0.32&0.56&0.25&0.35&0.17&0.25\\
&&&&&&\\
\textbf{Wealth measures}&&&&&&\\
\hspace*{2em}Finished floor&0.30&0.40&0.44&0.53&0.52&0.61\\
\hspace*{2em}Electricity&0.50&0.60&0.78&0.87&0.93&0.96\\
\hspace*{2em}Owns radio&0.24&0.34&0.05&0.08&0.04&0.05\\
\hspace*{2em}Owns TV&0.27&0.45&0.46&0.66&0.50&0.70\\
\hspace*{2em}Owns refrigerator&0.06&0.13&0.17&0.32&0.25&0.43\\
\hspace*{2em}Owns bicycle&0.43&0.62&0.46&0.59&0.45&0.55\\
\hspace*{2em}Owns car&0.01&0.03&0.03&0.06&0.04&0.09\\
\hspace*{2em}Uses toilet/latrine&0.30&0.41&0.48&0.58&0.71&0.81\\
\hspace*{2em}Owns land&.&.&0.30&0.28&0.37&0.31\\
&&&&&&\\
\textbf{N}&1,769&2,630&7,747&15,695&6,658&14,226\\
&&&&&&\\
\textbf{Women who gave birth 3--12 months before the survey}&&&&&&\\
&&&&&&\\
\textbf{Healthcare measures}&&&&&&\\
\hspace*{2em}Birth in a health facility&0.34&0.45&0.76&0.85&0.86&0.93\\
\hspace*{2em}4+ antenatal visits&0.31&0.40&0.44&0.54&0.53&0.61\\
&&&&&&\\
\textbf{Wealth measures}&&&&&&\\
\hspace*{2em}Finished floor&0.32&0.42&0.43&0.53&0.48&0.60\\
\hspace*{2em}Electricity&0.53&0.64&0.77&0.87&0.93&0.97\\
\hspace*{2em}Owns radio&0.22&0.35&0.05&0.08&0.03&0.04\\
\hspace*{2em}Owns TV&0.28&0.48&0.46&0.67&0.48&0.70\\
\hspace*{2em}Owns refrigerator&0.06&0.16&0.15&0.31&0.22&0.42\\
\hspace*{2em}Owns bicycle&0.45&0.62&0.45&0.57&0.45&0.55\\
\hspace*{2em}Owns car&0.01&0.03&0.02&0.06&0.04&0.08\\
\hspace*{2em}Uses toilet/latrine&0.34&0.40&0.46&0.57&0.68&0.81\\
\hspace*{2em}Owns land&.&.&0.32&0.30&0.36&0.32\\
&&&&&&\\
\textbf{N}&3,043&4,205&13,477&24,067&11,073&21,707\\
&&&&&&\\
\bottomrule
\end{tabular}


        \begin{tablenotes}
            \item put table notes here
        \end{tablenotes}
    \end{threeparttable}
\end{table}
\end{landscape}


\begin{landscape}

\begin{table}[!htbp]
    \centering
    \begin{threeparttable}[t]
        \caption{:  Summary statistics for variables used in mediation analysis}
        \label{tab:sumstats}

        \scriptsize
        \setlength{\tabcolsep}{6pt}
        \renewcommand{\arraystretch}{1}

        \begin{tabular}{l*{4}{>{\centering\arraybackslash}p{3.2cm}}}
\toprule
 & \multicolumn{2}{c}{sample: currently pregnant women} & \multicolumn{2}{c}{sample: women who gave birth 3+ months before the survey} \\\\
\cmidrule(lr){2-3}\cmidrule(lr){4-5}
 & no say in own healthcare & no say in visits to family/friends & gave birth in a health facility & had 4+ antenatal visits \\\\
\midrule
\textbf{NFHS-3}&&&&\\
\hspace*{2em}no controls&0.184*** \newline (0.013)&0.241*** \newline (0.013)&0.105*** \newline (0.006)&0.070*** \newline (0.006)\\
\hspace*{2em}wealth controls&0.191*** \newline (0.013)&0.242*** \newline (0.013)&0.035*** \newline (0.005)&0.005 \newline (0.005)\\
&&&&\\
\textbf{NFHS-4}&&&&\\
\hspace*{2em}no controls&0.069*** \newline (0.012)&0.101*** \newline (0.012)&0.088*** \newline (0.002)&0.076*** \newline (0.003)\\
\hspace*{2em}wealth controls&0.079*** \newline (0.012)&0.106*** \newline (0.012)&0.045*** \newline (0.002)&0.012*** \newline (0.002)\\
&&&&\\
\textbf{NFHS-5}&&&&\\
\hspace*{2em}no controls&0.061*** \newline (0.013)&0.060*** \newline (0.013)&0.061*** \newline (0.002)&0.059*** \newline (0.003)\\
\hspace*{2em}wealth controls&0.070*** \newline (0.013)&0.069*** \newline (0.013)&0.034*** \newline (0.002)&0.016*** \newline (0.003)\\
\bottomrule
\end{tabular}


        \begin{tablenotes}
            \item put table notes here
        \end{tablenotes}
    \end{threeparttable}
\end{table}
\end{landscape}

\bibliographystyle{apalike}
\bibliography{references}
\end{document}
