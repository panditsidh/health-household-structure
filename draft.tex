\documentclass[12pt]{article}
\usepackage{graphicx} %package to manage images
\graphicspath{ {./figures/} }
\usepackage{caption}
\usepackage[font=scriptsize]{subcaption}
\captionsetup[figure]{labelsep=none}
\captionsetup[table]{labelsep=none}
\usepackage{bbm}
\usepackage{amsmath}
\usepackage{import}
\usepackage{array}
\usepackage{booktabs}     % no extra vertical space between paragraphs
\usepackage{indentfirst}
\usepackage{afterpage}
\usepackage{floatrow}
\usepackage{pdflscape}
\usepackage{soul}
\usepackage{float}
\usepackage{adjustbox}
\usepackage{longtable}
\usepackage{caption}
\usepackage{setspace}
\usepackage{afterpage}
\usepackage[margin=1in]{geometry}
\usepackage[round]{natbib}
\usepackage{hyperref}
\usepackage{titlesec}
\usepackage{threeparttable}
\usepackage{setspace}
\usepackage{float}
\floatstyle{plaintop}
\restylefloat{table}
\usepackage{rotating}
\usepackage{authblk}
\usepackage{makecell}


% Optional styling for nicer author/affil typography
\renewcommand\Authfont{\normalsize\bfseries}
\renewcommand\Affilfont{\small\normalfont}
\setlength{\affilsep}{0.3em} % space between author and affiliation

\setlength{\parindent}{0pt}   % standard indent (~0.25 inch)
\setlength{\parskip}{1em}       % no extra vertical space between paragraphs

\titleformat{\section}
  {\normalfont\bfseries\fontsize{14}{16}\selectfont}  % Bold 14pt
  {}{0pt}{}  % No numbering

\titleformat{\subsection}
  {\normalfont\bfseries\fontsize{12}{14}\selectfont}  % Bold 12pt
  {}{0pt}{}  % No numbering

\titleformat{\subsubsection}
  {\normalfont\itshape\fontsize{12}{14}\selectfont}  % Italic 12pt
  {}{0pt}{}  % No numbering
  
\begin{document}
\title{Pregnant women's health and health decision making in Indian households}
\date{\today}
\maketitle

\section{Introduction}

Health during pregnancy matters for both women and their children.  The ability to make decisions about your own healthcare may have positive consequences for health and is a good in itself.  

Both theory (modernization theory) and evidence (Allendorf, 2012) suggest a trend towards nuclear households in India.  Scholars have long associated nuclear households with more autonomy for women (cite Jeffreys, others).  

This paper investigates trends in household structure among pregnant women in India in 3 rounds of the NFHS.  Contrary to the modernization literature, we find an increasing proportion of pregnant women live in patrilocal joint households, and that a shrinking proportion live in nuclear households.  The proportion living in natal households remained stable between 2005 and 2020.  

The primary contribution of this paper is to investigate the consequences of this demographic shift for pregnant women's health and health care decision making.  We further point researchers interested in understanding the causes of increasing co-residence of pregnant women with their in-laws towards some possible candidate explanations.

The increase in joint patrilocal households seems not to have had negative consequences for women's decision making about their own health care.  Pregnant women's autonomy as measured by their say in healthcare decisions has increased significantly in all household types, and gaps have narrowed. Women in joint patrilocal are now only x\% pp less likely to report say in their own decisions than women in nuclear households, compared to X\% in 2005.

Consistent with the prior literature which studied trends in the 1990s and early 2000s, we find some health advantages of joint patrilocal living: women in joint patrilocal households have greater dairy consumption and are less likely to experience physical violence by their husbands.  (BMI?)  Another difference, which we investigate in further detail, is that recently delivered women living in joint patrilocal households are more likely to give birth in private facilities than those living in nuclear households.  This is consistent with them living in wealthier households.

Despite families' likely assumptions that more costly care is better care, joint families' choice of private providers may actually be harmful for pregnant women and children in parts of the country where private providers have worse outcomes than public providers.  (summary of an analysis about neonatal or early neonatal death regressed on household type for different regions, control for facility type, see if it disappears.  Try the same for csection in south...)

Given that the demographic trends that appear to be contributing to an increase in joint family living for pregnant women seem likely to continue (cite the other paper), and that nuclear families will get richer and be more likely to be able to afford private care, a policy recommendation is to positively norm pregnant women being capable of making their own decisions, while educating everyone in pregnant women's households about what good delivery care looks like (qualified provider, less intervention).
Health during pregnancy matters for both women and their children.  The ability to make decisions about your own healthcare may have positive consequences for health and is a good in itself.  

Both theory (modernization theory) and evidence (Allendorf, 2012) suggest a trend towards nuclear households in India.  Scholars have long associated nuclear households with more autonomy for women (cite Jeffreys, others).  

This paper investigates trends in household structure among pregnant women in India in 3 rounds of the NFHS.  Contrary to the modernization literature, we find an increasing proportion of pregnant women live in patrilocal joint households, and that a shrinking proportion live in nuclear households.  The proportion living in natal households remained stable between 2005 and 2020.  

The primary contribution of this paper is to investigate the consequences of this demographic shift for pregnant women's health and health care decision making.  We further point researchers interested in understanding the causes of increasing co-residence of pregnant women with their in-laws towards some possible candidate explanations.

The increase in joint patrilocal households seems not to have had negative consequences for women's decision making about their own health care.  Pregnant women's autonomy as measured by their say in healthcare decisions has increased significantly in all household types, and gaps have narrowed. Women in joint patrilocal are now only x\% pp less likely to report say in their own decisions than women in nuclear households, compared to X\% in 2005.

Consistent with the prior literature which studied trends in the 1990s and early 2000s, we find some health advantages of joint patrilocal living: women in joint patrilocal households have greater dairy consumption and are less likely to experience physical violence by their husbands.  (BMI?)  Another difference, which we investigate in further detail, is that recently delivered women living in joint patrilocal households are more likely to give birth in private facilities than those living in nuclear households.  This is consistent with them living in wealthier households.

Despite families' likely assumptions that more costly care is better care, joint families' choice of private providers may actually be harmful for pregnant women and children in parts of the country where private providers have worse outcomes than public providers.  (summary of an analysis about neonatal or early neonatal death regressed on household type for different regions, control for facility type, see if it disappears.  Try the same for csection in south...)

Given that the demographic trends that appear to be contributing to an increase in joint family living for pregnant women seem likely to continue (cite the other paper), and that nuclear families will get richer and be more likely to be able to afford private care, a policy recommendation is to positively norm pregnant women being capable of making their own decisions, while educating everyone in pregnant women's households about what good delivery care looks like (qualified provider, less intervention).


\section{Background}

\subsection{The significance of women's healthcare decision making}
Women's autonomy or decision-making power within their households has a significant effect on their healthcare, especially when it comes to neonatal health and maternal healthcare. Coffey et al.\ \cite{coffey2022mothers} show that children of lower-ranking mothers are more likely to die in early life. In joint patrilocal households in rural India women who are married to the younger brothers are assigned lower rank than women married to the older brother in the same household. As numerous healthcare policies are rooted in women empowerment evidence from this paper indicates the role of social ranking in empowerment.


\subsection{Women's health, autonomy, and household structure: Are there causal relationships?}

\subsection{The changing experience of pregnancy}

\subsection{Trends in the use of healthcare at birth}

\section{Data and Methods}

Describe the NFHS data.

\section{Results}

\section{Discussion}

\newpage

\section{References}

\bibliographystyle{apalike}
\bibliography{writing/references}


\newpage


\section{Tables and figures}


\section{Results}

\begin{itemize}

    \item \textbf{Figure 1: Distribution of household structure across all NFHS rounds.}  
    This establishes the broader demographic context: nuclear, natal, and patrilocal households across five survey rounds.

    \item \textbf{Table 1: Household structure by subgroup (Allendorf-style).}  
    We show the percent of pregnant women living in patrilocal households by subgroup (caste, religion, parity, education, region) for NFHS-3, NFHS-4, and NFHS-5.  
    This demonstrates that the increase in patrilocal living is \emph{broad-based} across nearly all groups.  
    (Here we can briefly mention the 30\% parity decomposition result to show that part of the change is compositional.)

    \item \textbf{Figure 2: Trends in key outcomes by household structure.}  
    Four panels: no say in healthcare, no say due to money, dairy daily consumption, and BMI.  
    These descriptive figures show:
    \begin{itemize}
        \item Women's autonomy is consistently \emph{lower} in patrilocal households, though gaps narrow over time.
        \item Money as a barrier to healthcare is most common in nuclear households.
        \item Dairy consumption is much higher in patrilocal households.
        \item BMI is highest among nuclear women despite lower dairy intake, suggesting complex pathways.
    \end{itemize}
    These patterns show that autonomy disadvantages coexist with material advantages in patrilocal households, motivating a regression-based approach.

    \item \textbf{Table 2: Descriptive statistics in NFHS-5 (Allendorf-style).}  
    For each outcome, we report means for: all women, patrilocal households, and nuclear households, with significance of differences.  
    This shows the cross-sectional inequalities by household structure in the most recent round.

    \item \textbf{Tables 3 and 4: Patrilocal--nuclear regression gaps (unadjusted and adjusted).}  
    Columns are NFHS-3, NFHS-4, and NFHS-5; rows are outcomes.  
    Table 3 includes only state fixed effects.  
    Table 4 adds controls (age, education, caste/religion, urban residence, and gestational duration where relevant).  
    These tables show:

    \begin{itemize}

        \item \textbf{Convergence over time:}  
        Coefficients shrink across rounds for almost all outcomes, indicating convergence in norms and living conditions across household structures.

        \item \textbf{Persistent autonomy disadvantage:}  
        Patrilocal women consistently have less say in their own healthcare and mobility.  
        Gaps narrow but remain large and statistically significant even after full controls, suggesting a structural feature of patrilocal living.

        \item \textbf{Stable patrilocal advantage in dairy consumption:}  
        Large, positive, and significant in all rounds; remains robust with controls.  
        Indicates persistent material advantages in patrilocal households.

        \item \textbf{Domestic violence consistently lower in patrilocal households:}  
        Physical violence is significantly lower for patrilocal women in all rounds; sexual violence is modestly lower.  m

        \item \textbf{Home birth gap persists but shrinks sharply:}  
        Patrilocal women are less likely to deliver at home; gaps decline substantially over time and with controls.  
        Suggests that patrilocal households facilitate institutional delivery (wealth, transport, social networks).

        \item \textbf{C-section rates higher in patrilocal households:}  
        Coefficients are consistently positive.  
        Even after controls, patrilocal women are more likely to have C-sections, consistent with higher use of private facilities.

        \item \textbf{BMI reversal after controls:}  
        Unadjusted: patrilocal women are thinner in later rounds.  
        Adjusted: patrilocal women have slightly \emph{higher} BMI.  
        Indicates that raw BMI gaps are compositional (education, wealth, caste).

        \item \textbf{Anemia differences disappear after controls:}  
        Unadjusted gaps favor patrilocal women, but become small or insignificant after controls.  
        Household structure has little independent association with anemia.

    \end{itemize}

    TODO TABLE 4: neonatal survival regressions

    Together, these patterns suggest that increasing patrilocality does \emph{not} uniformly worsen women's health.  Patrilocal living is associated with lower autonomy but better material conditions. Over time, as nuclear and patrilocal households converge in wealth and norms, gaps shrink across nearly all outcomes.

    

\end{itemize}

    
    
\begin{figure}[H]
    \centering
    \includegraphics[width=\textwidth]{figures/Money is a problem for healthcare access_pregnant.png}
\end{figure}


\begin{table}[H]
    \centering
    \begin{threeparttable}[t]
        \caption{: Percentage of young married women living in patrilocal extended families as opposed to nuclear families}
        \label{tab:unadjusted}

        
        \setlength{\tabcolsep}{3pt}
        \renewcommand{\arraystretch}{1.2}

        \begin{tabular}{lcccc}
\toprule
Category & NFHS-3 & NFHS-4 & NFHS-5 & Sig. \\\\
\midrule
\textbf{Region}&&&&\\
\hspace{1em}focus&45.8&47.8&49.0&***\\
\hspace{1em}central&46.7&51.8&54.4&***\\
\hspace{1em}east&43.7&44.8&42.5&\\
\hspace{1em}west&47.0&49.0&53.0&***\\
\hspace{1em}north&51.6&61.9&60.7&***\\
\hspace{1em}south&34.0&36.8&39.7&***\\
\hspace{1em}northeast&37.0&40.3&47.9&***\\
&&&&\\
\textbf{Social group}&&&&\\
\hspace{1em}Forward Caste&40.5&44.6&47.4&***\\
\hspace{1em}OBC&40.8&44.9&47.0&***\\
\hspace{1em}Dalit&43.4&47.6&50.7&***\\
\hspace{1em}Adivasi&48.3&49.0&50.8&**\\
\hspace{1em}Muslim&42.3&43.1&42.5&\\
\hspace{1em}Sikh, Jain, Christian&57.5&45.5&48.5&**\\
&&&&\\
\textbf{Residence}&&&&\\
\hspace{1em}urban&39.2&41.2&44.8&***\\
\hspace{1em}rural&45.3&48.6&49.6&***\\
&&&&\\
\textbf{Age group}&&&&\\
\hspace{1em}15-19&51.3&45.5&46.5&***\\
\hspace{1em}20-24&53.9&58.0&60.8&***\\
\hspace{1em}25-29&44.2&52.1&55.6&***\\
\hspace{1em}30-34&32.3&40.8&42.4&***\\
\hspace{1em}35-39&25.9&28.0&33.3&***\\
\hspace{1em}40-44&14.6&19.1&24.0&***\\
\hspace{1em}45-49&13.6&15.2&14.9&\\
&&&&\\
\bottomrule
\end{tabular}

                
        \begin{tablenotes}
            \item sig column is the significance of the difference between nfhs-5 and nfhs-3
        \end{tablenotes}
    \end{threeparttable}
\end{table}



\begin{table}[H]
    \centering
    \begin{threeparttable}[t]
        \caption{: Coefficient on patrilocal indicator from a regression using only state fixed effects}
        \label{tab:unadjusted}

        
        \setlength{\tabcolsep}{3pt}
        \renewcommand{\arraystretch}{1.2}

        Table 2. Patrilocal–Nuclear Gaps without controls by NFHS Round
\begin{tabular}{lccc}
\hline
Outcome & 2005-06 & 2015-16 & 2019-21 \\\\
\hline
Consumes meat/egg/fish at least weekly&-0.002&0.000&-0.006\\
Consumes dairy daily&0.121***&0.084***&0.082***\\
No say in own healthcare&0.155***&0.061***&0.065***\\
No say in family visits&0.204***&0.084***&0.064***\\
Any anemia&-0.041***&-0.027***&-0.024***\\
Body mass index (BMI)&0.000&-0.333***&-0.509***\\
Physical domestic violence&-0.078***&-0.045***&-0.038***\\
Sexual domestic violence&-0.011&-0.021***&-0.020***\\
Home birth (3–12 months ago)&-0.125***&-0.095***&-0.055***\\
C-section (3–12 months ago)&0.037***&0.036***&0.055***\\
\hline
\end{tabular}

                
        \begin{tablenotes}
            \item notes on sample for national vars
            \item notes on sample for state vars
            \item notes on sample for dv vars
            \item notes on sample for last birth vars
        \end{tablenotes}
    \end{threeparttable}
\end{table}



\begin{table}[H]
    \centering
    \begin{threeparttable}[t]
        \caption{: Coefficient on patrilocal indicator from a regression using controls}
        \label{tab:unadjusted}

        
        \setlength{\tabcolsep}{3pt}
        \renewcommand{\arraystretch}{1.2}

        \begin{tabular}{lccc}
\hline
Outcome & 2005-06 & 2015-16 & 2019-21 \\\\
\hline
Consumes meat/egg/fish at least weekly&-0.004&0.005&-0.001\\
Consumes dairy daily&0.064***&0.047***&0.044***\\
No say in own healthcare&0.133***&0.046***&0.053***\\
No say in family visits&0.166***&0.076***&0.054***\\
Any anemia&-0.026&-0.019**&-0.005\\
Body mass index (BMI)&0.285**&0.187**&0.077\\
Physical domestic violence&-0.044***&-0.023*&-0.018\\
Sexual domestic violence&-0.008&-0.013*&-0.011*\\
Home birth (3–12 months ago)&-0.053***&-0.047***&-0.017\\
C-section (3–12 months ago)&0.011&0.019&0.035**\\
\hline
\end{tabular}

                
        \begin{tablenotes}
            \item controls: parity(1,2,3,4+), age 5 year bins, education attainment, social group, whether distance to facility is a barrier in accessing healthcare
            \item notes on sample for national vars
            \item notes on sample for state vars
            \item notes on sample for dv vars
            \item notes on sample for last birth vars
        \end{tablenotes}
    \end{threeparttable}
\end{table}


WTS:
how coefs change over time
how they change after controls


maybe add
- getting permission problem to accessing healthcare (check this)
- full antenatal care
- underweight
- public birth
- private birth
- last birth neonatal death



\begin{figure}[H]
    \centering
    \includegraphics[width=\textwidth]{figures/hhstruc pregnant.png}
\end{figure}

\begin{figure}[H]
    \centering
    \includegraphics[width=\textwidth]{figures/No say in own healthcare_pregnant.png}

    \parbox{1\linewidth}{\footnotesize Notes: NFHS-2 asks "who decides on obtaining health care". NFHS-3 asks "Final say on own health care". NFHS-4/5 asks "person who usually decides on respondent's healthcare"}
\end{figure}


% \begin{figure}[H]
%     \centering
%     \includegraphics[width=\textwidth]{figures/No say in large purchases_pregnant.png}
%     \parbox{1\linewidth}{\footnotesize Notes: NFHS-2 asks "Who decides to purchase jewelry". NFHS-3 asks "Final say on making large household purchases". NFHS-4/5 asks "Person who usually decides on large household purchases"}
% \end{figure}


% \begin{figure}[H]
%     \centering
%     \includegraphics[width=\textwidth]{figures/Experienced physical domestic violence.png}
%     \parbox{1\linewidth}{\footnotesize Notes: NFHS-2 asks "has been beaten since age 15, and husband has beaten respondent". NFHS-3/4/5 asks "spouse ever pushed/slapped/punched etc.}
% \end{figure}

% \begin{figure}[H]
%     \centering
%     \includegraphics[width=\textwidth]{figures/Consumes dairy daily_pregnant.png}
% \end{figure}



% \begin{figure}[H]
%     \centering
%     \includegraphics[width=\textwidth]{figures/hhstruc_bmi_preg.png}
% \end{figure}


% \begin{figure}[H]
%     \centering
%     \includegraphics[width=\textwidth]{figures/Wealth index z score.png}
% \end{figure}

% \begin{figure}[H]
%     \centering
%     \includegraphics[width=\textwidth]{figures/Delivered by Csection last 3-12 mo._nonpregnant.png}
% \end{figure}

% \begin{figure}[H]
%     \centering
%     \includegraphics[width=\textwidth]{figures/Last birth occurred at home within 3-12 mo._nonpregnant.png}
% \end{figure}

% \begin{figure}[H]
%     \centering
%     \includegraphics[width=\textwidth]{figures/region facility round2.png}
% \end{figure}

% \begin{figure}[H]
%     \centering
%     \includegraphics[width=\textwidth]{figures/region facility round3.png}
% \end{figure}

% \begin{figure}[H]
%     \centering
%     \includegraphics[width=\textwidth]{figures/region facility round4.png}
% \end{figure}

\begin{figure}[H]
    \centering
    \includegraphics[width=\textwidth]{figures/region facility round5.png}
\end{figure}




\section{Appendix}



% \begin{figure}[H]
%     \centering
%     \includegraphics[width=\textwidth]{figures/hhstruc non pregnant.png}
% \end{figure}

% \begin{table}[H]
%     \centering
%     \setlength{\tabcolsep}{4pt} % shrink column padding
%     \footnotesize % shrink text
%     \caption{: Household structure 3+ mopreg women are observed in by subgroup}
%     \label{tab:sumstat}
%     \begin{adjustbox}{width=\textwidth}
%         \begin{tabular}{lcccc}
\toprule
Group & NFHS-2 & NFHS-3 & NFHS-4 & NFHS-5 \\\\
\midrule
\textbf{Nuclear Households}&&&&\\
India&29.4 (28.0, 30.9)&32.9 (31.1, 34.7)&26.1 (25.4, 26.9)&24.6 (23.9, 25.3)\\
Rural&29.6 (26.7, 32.6)&33.7 (30.8, 36.6)&28.1 (26.3, 29.9)&26.5 (24.8, 28.3)\\
Urban&29.4 (27.7, 31.0)&32.6 (30.5, 34.8)&25.4 (24.6, 26.1)&23.9 (23.1, 24.7)\\
Forward Caste&36.5 (31.5, 41.5)&40.2 (35.1, 45.2)&30.5 (28.6, 32.5)&29.7 (27.0, 32.4)\\
OBC&32.0 (28.7, 35.3)&38.8 (35.1, 42.6)&28.4 (26.7, 30.0)&25.9 (24.4, 27.4)\\
Dalit&26.8 (24.2, 29.4)&29.9 (27.0, 32.9)&23.7 (22.6, 24.8)&20.9 (19.8, 22.1)\\
Adivasi&25.0 (22.2, 27.7)&24.0 (20.7, 27.3)&20.2 (18.2, 22.2)&19.9 (17.8, 21.9)\\
Muslim&33.2 (29.3, 37.1)&38.2 (33.2, 43.2)&31.7 (29.8, 33.7)&32.0 (30.0, 34.1)\\
Sikh, Jain, Christian&25.8 (17.2, 34.4)&20.2 (11.9, 28.6)&18.5 (13.3, 23.6)&17.3 (11.1, 23.6)\\
.&&&&\\
\textbf{Sasural Households}&&&&\\
India&54.3 (52.8, 55.8)&51.7 (49.9, 53.5)&58.5 (57.6, 59.3)&59.9 (59.1, 60.8)\\
Rural&53.2 (50.1, 56.3)&51.8 (48.7, 54.9)&56.0 (53.9, 58.1)&57.8 (56.0, 59.7)\\
Urban&54.6 (52.9, 56.4)&51.7 (49.6, 53.9)&59.5 (58.6, 60.3)&60.7 (59.7, 61.6)\\
Forward Caste&51.2 (46.3, 56.1)&46.3 (41.1, 51.5)&55.9 (53.7, 58.1)&58.3 (55.6, 60.9)\\
OBC&54.4 (51.0, 57.8)&48.5 (44.8, 52.2)&57.1 (55.3, 58.9)&58.6 (56.9, 60.2)\\
Dalit&55.1 (52.3, 58.0)&52.3 (48.9, 55.6)&60.3 (59.1, 61.6)&62.5 (61.2, 63.8)\\
Adivasi&57.4 (54.3, 60.4)&60.7 (57.0, 64.4)&64.8 (62.6, 67.1)&64.6 (61.9, 67.2)\\
Muslim&49.1 (45.0, 53.2)&47.0 (42.5, 51.5)&51.6 (49.4, 53.8)&53.1 (51.0, 55.3)\\
Sikh, Jain, Christian&62.8 (52.2, 73.4)&64.6 (54.0, 75.1)&65.4 (59.0, 71.8)&62.4 (54.6, 70.1)\\
.&&&&\\
\textbf{Natal Households}&&&&\\
India&16.3 (15.1, 17.5)&15.4 (14.1, 16.6)&15.4 (14.8, 16.0)&15.5 (14.8, 16.1)\\
Rural&17.2 (14.7, 19.6)&14.5 (12.3, 16.7)&15.9 (14.5, 17.3)&15.6 (14.2, 17.0)\\
Urban&16.0 (14.7, 17.4)&15.7 (14.2, 17.1)&15.2 (14.5, 15.8)&15.4 (14.7, 16.1)\\
Forward Caste&12.3 (9.4, 15.1)&13.5 (10.1, 17.0)&13.5 (12.0, 15.1)&12.0 (10.5, 13.5)\\
OBC&13.6 (11.1, 16.1)&12.7 (10.3, 15.0)&14.5 (13.3, 15.8)&15.5 (14.3, 16.7)\\
Dalit&18.0 (15.6, 20.5)&17.8 (15.4, 20.1)&16.0 (15.0, 16.9)&16.5 (15.5, 17.6)\\
Adivasi&17.7 (15.3, 20.1)&15.2 (12.5, 18.0)&14.9 (13.2, 16.6)&15.5 (13.1, 18.0)\\
Muslim&17.7 (14.5, 20.9)&14.7 (11.5, 18.0)&16.7 (15.0, 18.4)&14.8 (13.2, 16.4)\\
Sikh, Jain, Christian&11.4 (5.2, 17.6)&15.2 (8.1, 22.4)&16.2 (11.4, 20.9)&20.3 (13.4, 27.2)\\
.&&&&\\
&&&&\\
\bottomrule
\end{tabular}

%     \end{adjustbox}
% \end{table}

% \begin{table}[!htbp]
%     \centering
%     \begin{threeparttable}[t]
%         \caption{:  Proportion of pregnant women who report having no say in own healthcare}
%         \label{tab:sumstats}

%         \scriptsize
%         \setlength{\tabcolsep}{2pt}
%         \renewcommand{\arraystretch}{1.2}

%         \begin{tabular}{l*{3}{>{\centering\arraybackslash}p{1.4cm}}}
\toprule
 & Nuclear & Patrilocal & Natal \\\\
\midrule
\textbf{NFHS-2 (1998–99)}&&&\\
\hspace*{2em}Adivasi&.57&.63&.58\\
\hspace*{2em}Dalit&.55&.56&.56\\
\hspace*{2em}OBC&.53&.57&.56\\
\hspace*{2em}Forward caste&.56&.56&.62\\
\hspace*{2em}Muslim&.55&.58&.60\\
&&&\\
\textbf{NFHS-3 (2005–06)}&&&\\
\hspace*{2em}Adivasi&.42&.58&.48\\
\hspace*{2em}Dalit&.33&.50&.51\\
\hspace*{2em}OBC&.35&.57&.53\\
\hspace*{2em}Forward caste&.31&.50&.49\\
\hspace*{2em}Muslim&.38&.55&.51\\
&&&\\
\textbf{NFHS-4 (2015–16)}&&&\\
\hspace*{2em}Adivasi&.25&.28&.24\\
\hspace*{2em}Dalit&.28&.33&.21\\
\hspace*{2em}OBC&.27&.35&.18\\
\hspace*{2em}Forward caste&.26&.33&.22\\
\hspace*{2em}Muslim&.23&.31&.18\\
&&&\\
\textbf{NFHS-5 (2019–21)}&&&\\
\hspace*{2em}Adivasi&.17&.21&.11\\
\hspace*{2em}Dalit&.20&.23&.20\\
\hspace*{2em}OBC&.20&.28&.17\\
\hspace*{2em}Forward caste&.16&.23&.17\\
\hspace*{2em}Muslim&.25&.31&.24\\
\bottomrule
\end{tabular}

                
%         \begin{tablenotes}
%             \item NFHS-2: who decides on obtaining healthcare
%             \item NFHS-3: final say on own healthcare
%             \item NFHS-4/5: person who usually decides on respondent's healthcare
%         \end{tablenotes}
%     \end{threeparttable}
% \end{table}


% \begin{figure}[H]
%     \centering
%     \includegraphics[width=\textwidth]{figures/No say in visiting natal family_pregnant.png}
%     \parbox{1\linewidth}{\footnotesize Notes: NFHS-2 asks "permission needed to visit relatives or friends. NFHS-3 asks "final say on visit to family or relatives". NFHS-4/5 asks "person who usually decides on visits to family or relatives}
% \end{figure}


% \begin{figure}[H]
%     \centering
%     \includegraphics[width=\textwidth]{figures/Consumes meat,egg,fish daily_pregnant.png}
% \end{figure}


% \begin{figure}[H]
%     \centering
%     \includegraphics[width=\textwidth]{figures/Any anemia (DHS cutoff for pregnancy)_pregnant.png}
% \end{figure}


% \begin{figure}[H]
%     \centering
%     \includegraphics[width=\textwidth]{figures/Experienced sexual domestic violence_pregnant.png}
% \end{figure}



\bibliographystyle{asa}
\bibliography{reference}
\end{document}
