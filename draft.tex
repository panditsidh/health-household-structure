\documentclass[12pt]{article}
\usepackage{graphicx} %package to manage images
\graphicspath{ {./figures/} }
\usepackage{caption}
\usepackage[font=scriptsize]{subcaption}
\captionsetup[figure]{labelsep=none}
\captionsetup[table]{labelsep=none}
\usepackage{bbm}
\usepackage{amsmath}
\usepackage{import}
\usepackage{array}
\usepackage{booktabs}     % no extra vertical space between paragraphs
\usepackage{indentfirst}
\usepackage{afterpage}
\usepackage{floatrow}
\usepackage{pdflscape}
\usepackage{soul}
\usepackage{float}
\usepackage{adjustbox}
\usepackage{longtable}
\usepackage{caption}
\usepackage{setspace}
\usepackage{afterpage}
\usepackage[margin=1in]{geometry}
\usepackage[round]{natbib}
\setcitestyle{authoryear}
\usepackage{hyperref}
\usepackage{titlesec}
\usepackage{threeparttable}
\usepackage{setspace}
\usepackage{float}
\floatstyle{plaintop}
\restylefloat{table}
\usepackage{rotating}
\usepackage{authblk}
\usepackage{makecell}


% Optional styling for nicer author/affil typography
\renewcommand\Authfont{\normalsize\bfseries}
\renewcommand\Affilfont{\small\normalfont}
\setlength{\affilsep}{0.3em} % space between author and affiliation

\setlength{\parindent}{0pt}   % standard indent (~0.25 inch)
\setlength{\parskip}{1em}       % no extra vertical space between paragraphs

\titleformat{\section}
  {\normalfont\bfseries\fontsize{14}{16}\selectfont}  % Bold 14pt
  {}{0pt}{}  % No numbering

\titleformat{\subsection}
  {\normalfont\bfseries\fontsize{12}{14}\selectfont}  % Bold 12pt
  {}{0pt}{}  % No numbering

\titleformat{\subsubsection}
  {\normalfont\itshape\fontsize{12}{14}\selectfont}  % Italic 12pt
  {}{0pt}{}  % No numbering
  
\begin{document}
\title{Pregnant women's health and health decision making in Indian households}
\date{\today}
\maketitle

\section{Introduction}

Health during pregnancy matters for both women and their children.  The ability to make decisions about your own healthcare may have positive consequences for health and is a good in itself.  

Both theory (modernization theory) and evidence (\cite{allendorf2013going}) suggest a trend towards nuclear households in India.  Scholars have long associated nuclear households with more autonomy for women (\cite{jacobson1992women};\cite{jeffery1989labour};\cite{madan1989family}).  

Research questions: What is the relationship between household structure and pregnant women's autonomy (say in decisions about own health, say in visits to family and friends, and getting permission is a problem)?  What is the relationship between household structure and pregnant women's health outcomes (Nutrition: dairy consumption, meat consumption, BMI, underweight, anemia; Violence: sexual, physical;  Pregnancy and birth care and outcomes: ANC visits, home birth, c-section, NNM, (rural north only): private hospital)?  

Short description of findings: Pregnant women in joint households have less say.  Pregnant women in joint households have better pregnancy health outcomes, which are mostly accounted for by wealth differences.

The increase in joint household living is not impacting autonomy much because there is a countervailing improvement in autonomy for everyone, and more so for women in joint households -- if joint HH had stayed the same, only 4\% more women would have say in health care and only x\% more women would have say in visits to family and friends.

The increase in joint household living is...[do a similar decomposition with our favorite outcomes]



**

This paper investigates trends in household structure among pregnant women in India in 3 rounds of the NFHS.  Contrary to the modernization literature, we find an increasing proportion of pregnant women live in patrilocal joint households, and that a shrinking proportion live in nuclear households.  The proportion living in natal households remained stable between 2005 and 2020.  

The primary contribution of this paper is to investigate the consequences of this demographic shift for pregnant women's health and health care decision making.  We further point researchers interested in understanding the causes of increasing co-residence of pregnant women with their in-laws towards some possible candidate explanations.

The increase in joint patrilocal households seems not to have had negative consequences for women's decision making about their own health care.  Pregnant women's autonomy as measured by their say in healthcare decisions has increased significantly in all household types, and gaps have narrowed. Women in joint patrilocal are now only x\% pp less likely to report say in their own decisions than women in nuclear households, compared to X\% in 2005.

Consistent with the prior literature which studied trends in the 1990s and early 2000s, we find some health advantages of joint patrilocal living: women in joint patrilocal households have greater dairy consumption and are less likely to experience physical violence by their husbands.  (BMI?)  Another difference, which we investigate in further detail, is that recently delivered women living in joint patrilocal households are more likely to give birth in private facilities than those living in nuclear households.  This is consistent with them living in wealthier households.

Despite families' likely assumptions that more costly care is better care, joint families' choice of private providers may actually be harmful for pregnant women and children in parts of the country where private providers have worse outcomes than public providers.  (summary of an analysis about neonatal or early neonatal death regressed on household type for different regions, control for facility type, see if it disappears.  Try the same for csection in south...)

Given that the demographic trends that appear to be contributing to an increase in joint family living for pregnant women seem likely to continue (cite the other paper), and that nuclear families will get richer and be more likely to be able to afford private care, a policy recommendation is to positively norm pregnant women being capable of making their own decisions, while educating everyone in pregnant women's households about what good delivery care looks like (qualified provider, less intervention).
Health during pregnancy matters for both women and their children.  The ability to make decisions about your own healthcare may have positive consequences for health and is a good in itself.  

Both theory (modernization theory) and evidence (Allendorf, 2012) suggest a trend towards nuclear households in India.  Scholars have long associated nuclear households with more autonomy for women (cite Jeffreys, others).  

This paper investigates trends in household structure among pregnant women in India in 3 rounds of the NFHS.  Contrary to the modernization literature, we find an increasing proportion of pregnant women live in patrilocal joint households, and that a shrinking proportion live in nuclear households.  The proportion living in natal households remained stable between 2005 and 2020.  

The primary contribution of this paper is to investigate the consequences of this demographic shift for pregnant women's health and health care decision making.  We further point researchers interested in understanding the causes of increasing co-residence of pregnant women with their in-laws towards some possible candidate explanations.

The increase in joint patrilocal households seems not to have had negative consequences for women's decision making about their own health care.  Pregnant women's autonomy as measured by their say in healthcare decisions has increased significantly in all household types, and gaps have narrowed. Women in joint patrilocal are now only x\% pp less likely to report say in their own decisions than women in nuclear households, compared to X\% in 2005.

Consistent with the prior literature which studied trends in the 1990s and early 2000s, we find some health advantages of joint patrilocal living: women in joint patrilocal households have greater dairy consumption and are less likely to experience physical violence by their husbands.  (BMI?)  Another difference, which we investigate in further detail, is that recently delivered women living in joint patrilocal households are more likely to give birth in private facilities than those living in nuclear households.  This is consistent with them living in wealthier households.

Despite families' likely assumptions that more costly care is better care, joint families' choice of private providers may actually be harmful for pregnant women and children in parts of the country where private providers have worse outcomes than public providers.  (summary of an analysis about neonatal or early neonatal death regressed on household type for different regions, control for facility type, see if it disappears.  Try the same for csection in south...)

Given that the demographic trends that appear to be contributing to an increase in joint family living for pregnant women seem likely to continue (cite the other paper), and that nuclear families will get richer and be more likely to be able to afford private care, a policy recommendation is to positively norm pregnant women being capable of making their own decisions, while educating everyone in pregnant women's households about what good delivery care looks like (qualified provider, less intervention).


\section{Background}

\subsection{The significance of women's healthcare decision making}
Women's autonomy or decision-making power within their households has a significant effect on their healthcare, especially when it comes to neonatal health and maternal healthcare. \cite{coffey2022mothers} show that children of lower-ranking mothers are more likely to die in early life. In joint patrilocal households in rural India women who are married to the younger brothers are assigned lower rank than women married to the older brother in the same household. As numerous healthcare policies are rooted in women empowerment evidence from this paper indicates the role of social ranking in empowerment. Further, \cite{coffey2021neonatal} emphasizes the role of decision making power in case of neonatal mortality through maternal under nutrition which is concentrated among younger women who are early in their careers of childbearing. The findings also indicate the relevance of social status of women, household structure, and demographic relations in health outcomes in the developing world. 


\subsection{Women's health, autonomy, and household structure: Are there causal relationships?}
There is an increasing trend in decision making power among women in the last three rounds of NFHS. But at the same time we did observe an increase in young women's residence in the patrilocal extended households unlike \cite{allendorf2013going} where she observed an increase in young married women's residence in nuclear households. But their work emphasized that contrary to literature young women living in the nuclear households do not have better health than those living in patrilocal extended families. Further, their work showed that women living in patrilocal residences had an overall advantage Although this paper also notes that women living in nuclear families are more likely to consume meat, fish and eggs on a weekly basis but there is an overall advantage for women living in patrilocal extended families. Although \cite{hou2013effect} work doesn't show a direct association between household structure and women's health and autonomy but their work indicates that the presence of influential males' have an opposite effect in uptake of maternal health service uptake. 

\subsection{The changing experience of pregnancy}
Trends in decision making power indicate increase in involvement of the women in different forms of decision making in the household including their own healthcare. Studies by \cite{ghose2017women} using Bangladesh DHS indicate that women who make decisions along with their spouses are more likely to use more number of maternal health services. Thus, indicating an increase as well as advantage in husband's involvement in reproductive care. 

\subsection{Trends in the use of healthcare at birth}
There has been a significant change in use of healthcare at birth in India. Though there has been an increase in uptake of maternal healthcare services, there has also been a doubling in cesarean deliveries. This growth is closely related to the simultaneous decline in the share of home deliveries without medical supervision and an increase in institutional deliveries spearheaded by contemporary cash transfer policies encouraging women to use prenatal and postnatal care in modern health centers. Along the same time the major rise in cesarean deliveries occurred in private clinics where it rose from 27.5 percent to 40.8 percent as opposed to government facilities where it remained 11 percent in 2015-16(\cite{guilmoto2019trends}. Other studies such as \cite{singh2021maternal}'s study using NFHS data shows that there is still a persisting difference in the full use of ANC (antenatal care) and SBA (skilled birth attendant) especially among adolescent mothers. They also observed that the likelihood of using these services also decreases with increase in birth order. In the context of utilizing maternal healthcare services \cite{pathak2010economic} observed economic status as one of the parameters of utilizing health services. Even \cite{allendorf2013going} observes the same as in her paper she identifies economic status as a mediator for the patrilocal residence advantage where women in patrilocal extended households are more likely to use antenatal care and delivery assistance as well as consume milk and curd weekly and are also less likely to report physical violence


\section{Data and Methods}

This paper uses data from the National Family Health Survey (NFHS) collected in 2005-2006, 2015-16 and 2019-21. It is a cross-sectional, large-scale, nationally representative survey of households and women of reproductive age. I have restricted the sample to only pregnant women as here the focus is on pregnant women's access and experience in healthcare. Further, I have constructed the household structures using information from the household roster which provides data on who else resides in the same household as the respondent. The analytical sample consists of 5911 pregnant women interviewed in 2005-2006; 32,428 pregnant women interviewed in 2015-16, and 28,408 pregnant women interviewed in 2019-21. 

\section{Results}




\section{Discussion}

\newpage




\newpage


\section{Tables and figures}


\section{Results}

\begin{figure}[H]
    \centering
    \includegraphics[width=\textwidth]{figures/hhstruc pregnant.png}
    \caption{: trends in household structure over time}
\end{figure}


\section{Autonomy}

\begin{figure}[H]
    \centering
    \includegraphics[width=\textwidth]{figures3_4_5/autonomy/Getting permission is a problem for healthcare access_pregnant.png}
    \caption{: Change in getting permission to access healthcare}
\end{figure}

\begin{figure}[H]
    \centering
    \includegraphics[width=\textwidth]{figures3_4_5/autonomy/Money is a problem for healthcare access_pregnant.png}
    \caption{:Money as a barrier to healthcare}
\end{figure}

\begin{figure}[H]
    \centering
    \includegraphics[width=\textwidth]{figures3_4_5/autonomy/No say in own healthcare_pregnant.png}
    \caption{: Having say in healthcare}
\end{figure}






\section{Nutrition}

\begin{figure}[H]
    \centering
    \includegraphics[width=\textwidth]{figures3_4_5/nutrition/dairy_daily_pregnant.png}
    \caption{:Daily dairy consumption}
\end{figure}

\begin{figure}[H]
    \centering
    \includegraphics[width=\textwidth]{figures3_4_5/nutrition/meat_egg_fish_daily_pregnant.png}
    \caption{:Consumption of egg, meat and fish}
\end{figure}


\begin{figure}[H]
    \centering
    \includegraphics[width=\textwidth]{figures3_4_5/nutrition/COEFS bmi .png}
    \caption{:BMI}
\end{figure}

\begin{figure}[H]
    \centering
    \includegraphics[width=\textwidth]{figures3_4_5/nutrition/anemic_pregnant.png}
    \caption{:anemic}
\end{figure}




\section{Healthcare at birth}

\begin{figure}[H]
    \centering
    \includegraphics[width=\textwidth]{figures3_4_5/healthcare at birth/COEFS home birth .png}
    \caption{:Birth at Home}
\end{figure}

\begin{figure}[H]
    \centering
    \includegraphics[width=\textwidth]{figures3_4_5/healthcare at birth/COEFS private birth .png}
    \caption{:Birth at private facility}
\end{figure}

\begin{figure}[H]
    \centering
    \includegraphics[width=\textwidth]{figures3_4_5/healthcare at birth/COEFS public birth .png}
    \caption{:Birth at public facility}
\end{figure}


\begin{itemize}

    \item \textbf{Figure 1: Distribution of household structure across all NFHS rounds.}  
    This establishes the broader demographic context: nuclear, natal, and patrilocal households across five survey rounds.

    \item \textbf{Table 1: Household structure by subgroup (Allendorf-style).}  
    We show the percent of pregnant women living in patrilocal households by subgroup (caste, religion, parity, education, region) for NFHS-3, NFHS-4, and NFHS-5.  
    This demonstrates that the increase in patrilocal living is \emph{broad-based} across nearly all groups.  
    (Here we can briefly mention the 30\% parity decomposition result to show that part of the change is compositional.)

    \item \textbf{Figure 2: Trends in key outcomes by household structure.}  
    Four panels: no say in healthcare, no say due to money, dairy daily consumption, and BMI.  
    These descriptive figures show:
    \begin{itemize}
        \item Women's autonomy is consistently \emph{lower} in patrilocal households, though gaps narrow over time.
        \item Money as a barrier to healthcare is most common in nuclear households.
        \item Dairy consumption is much higher in patrilocal households.
        \item BMI is highest among nuclear women despite lower dairy intake, suggesting complex pathways.
    \end{itemize}
    These patterns show that autonomy disadvantages coexist with material advantages in patrilocal households, motivating a regression-based approach.

    \item \textbf{Table 2: Descriptive statistics in NFHS-5 (Allendorf-style).}  
    For each outcome, we report means for: all women, patrilocal households, and nuclear households, with significance of differences.  
    This shows the cross-sectional inequalities by household structure in the most recent round.

    \item \textbf{Tables 3 and 4: Patrilocal--nuclear regression gaps (unadjusted and adjusted).}  
    Columns are NFHS-3, NFHS-4, and NFHS-5; rows are outcomes.  
    Table 3 includes only state fixed effects.  
    Table 4 adds controls (age, education, caste/religion, urban residence, and gestational duration where relevant).  
    These tables show:

    \begin{itemize}

        \item \textbf{Convergence over time:}  
        Coefficients shrink across rounds for almost all outcomes, indicating convergence in norms and living conditions across household structures.

        \item \textbf{Persistent autonomy disadvantage:}  
        Patrilocal women consistently have less say in their own healthcare and mobility.  
        Gaps narrow but remain large and statistically significant even after full controls, suggesting a structural feature of patrilocal living.

        \item \textbf{Stable patrilocal advantage in dairy consumption:}  
        Large, positive, and significant in all rounds; remains robust with controls.  
        Indicates persistent material advantages in patrilocal households.

        \item \textbf{Domestic violence consistently lower in patrilocal households:}  
        Physical violence is significantly lower for patrilocal women in all rounds; sexual violence is modestly lower.  m

        \item \textbf{Home birth gap persists but shrinks sharply:}  
        Patrilocal women are less likely to deliver at home; gaps decline substantially over time and with controls.  
        Suggests that patrilocal households facilitate institutional delivery (wealth, transport, social networks).

        \item \textbf{C-section rates higher in patrilocal households:}  
        Coefficients are consistently positive.  
        Even after controls, patrilocal women are more likely to have C-sections, consistent with higher use of private facilities.

        \item \textbf{BMI reversal after controls:}  
        Unadjusted: patrilocal women are thinner in later rounds.  
        Adjusted: patrilocal women have slightly \emph{higher} BMI.  
        Indicates that raw BMI gaps are compositional (education, wealth, caste).

        \item \textbf{Anemia differences disappear after controls:}  
        Unadjusted gaps favor patrilocal women, but become small or insignificant after controls.  
        Household structure has little independent association with anemia.

    \end{itemize}

    TODO TABLE 4: neonatal survival regressions

    Together, these patterns suggest that increasing patrilocality does \emph{not} uniformly worsen women's health.  Patrilocal living is associated with lower autonomy but better material conditions. Over time, as nuclear and patrilocal households converge in wealth and norms, gaps shrink across nearly all outcomes.

    

\end{itemize}





\section{Appendix}

    
\begin{table}[H]
    \centering
    \begin{threeparttable}[t]
        \caption{: Descriptive stats}
        \label{tab:unadjusted}

        \scriptsize
        \setlength{\tabcolsep}{3pt}
        \renewcommand{\arraystretch}{1}

        \begin{tabular}{lcccc}
\toprule
Outcome & All & Nuclear & Patrilocal & Sig. \\\\
\midrule
&&&&\\
\textbf{Health}&&&&\\
    BMI (residualized)&0.038&0.406&-0.188&***\\
    Underweight&0.117&0.113&0.119&\\
    Any anemia&0.580&0.595&0.570&***\\
&&&&\\
\textbf{Diet}&&&&\\
    Consumes meat/egg/fish weekly&0.456&0.499&0.429&***\\
    Consumes dairy daily&0.522&0.465&0.557&***\\
&&&&\\
\textbf{Healthcare access}&&&&\\
    Distance is a problem (facility)&0.607&0.624&0.596&***\\
    Money is a problem (healthcare)&0.534&0.562&0.517&***\\
    4+ ANC visits&0.538&0.487&0.566&***\\
&&&&\\
\textbf{Region}&&&&\\
    North region&0.117&0.085&0.136&***\\
    Central region&0.084&0.077&0.088&***\\
    East region&0.137&0.160&0.123&***\\
    West region&0.143&0.126&0.154&***\\
    South region&0.170&0.193&0.155&***\\
    Northeast region&0.037&0.045&0.032&***\\
&&&&\\
\textbf{Social group}&&&&\\
    Forward caste&0.099&0.104&0.095&*\\
    OBC&0.225&0.233&0.220&*\\
    Dalit&0.357&0.327&0.375&***\\
    Adivasi&0.157&0.147&0.162&**\\
    Muslim&0.150&0.178&0.133&***\\
    Sikh/Jain/Christian&0.012&0.010&0.013&**\\
&&&&\\
\textbf{Education}&&&&\\
    No education&0.210&0.320&0.142&***\\
    Primary education&0.119&0.158&0.095&***\\
    Secondary education&0.518&0.427&0.573&***\\
    Higher education&0.154&0.095&0.190&***\\
&&&&\\
\textbf{Wealth}&&&&\\
    Poorest quintile&0.216&0.294&0.168&***\\
    Poorer quintile&0.213&0.224&0.207&**\\
    Middle quintile&0.199&0.183&0.208&***\\
    Richer quintile&0.196&0.163&0.216&***\\
    Richest quintile&0.176&0.136&0.201&***\\
&&&&\\
\textbf{Assets}&&&&\\
    Owns fridge&0.378&0.289&0.433&***\\
    Owns television&0.663&0.576&0.717&***\\
    Owns motorcycle&0.568&0.425&0.656&***\\
&&&&\\
\textbf{Parity}&&&&\\
    Parity: 1 child&0.281&0.215&0.321&***\\
    Parity: 2 children&0.231&0.286&0.197&***\\
    Parity: 3 children&0.119&0.183&0.080&***\\
    Parity: 4+ children&0.369&0.316&0.402&***\\
&&&&\\
\textbf{Decision making}&&&&\\
    No say in own healthcare&0.218&0.176&0.246&***\\
    Permission is a problem&0.381&0.366&0.392&***\\
    No say in family visits&0.219&0.175&0.248&***\\
    No say in purchases&0.243&0.183&0.284&***\\
&&&&\\
\textbf{Domestic violence}&&&&\\
    Physical domestic violence&0.069&0.089&0.055&***\\
    Sexual domestic violence&0.026&0.036&0.018&**\\
\bottomrule
\end{tabular}

                
        \begin{tablenotes}
            \item sample now is only patrilocal and nuclear
        \end{tablenotes}
    \end{threeparttable}
\end{table}



\begin{landscape}


\begin{table}[H]
    \centering
    \begin{threeparttable}[t]
        \caption{: patrilocal coeff}
        \label{tab:unadjusted}

        \scriptsize
        \setlength{\tabcolsep}{3pt}
        \renewcommand{\arraystretch}{1.7}

        \begin{tabular}{lccc}
\toprule
Outcome & 2005-06 & 2015-16 & 2019-21 \\\\
\midrule
\textbf{Panel A. Patrilocal coefficient without controls}&&&\\
    Consumes meat/egg/fish at least weekly&-0.0016 (-0.0271,  0.0238)& 0.0003 (-0.0143,  0.0149)&-0.0063 (-0.0209,  0.0083)\\
    Consumes dairy daily& 0.1215 ( 0.0898,  0.1532)***& 0.0845 ( 0.0697,  0.0993)***& 0.0815 ( 0.0668,  0.0962)***\\
    No say in own healthcare& 0.1546 ( 0.1268,  0.1825)***& 0.0615 ( 0.0340,  0.0890)***& 0.0648 ( 0.0375,  0.0922)***\\
    No say in family visits& 0.2041 ( 0.1769,  0.2312)***& 0.0845 ( 0.0565,  0.1125)***& 0.0641 ( 0.0368,  0.0914)***\\
    Any anemia&-0.0365 (-0.0674, -0.0056)**&-0.0231 (-0.0387, -0.0075)***&-0.0197 (-0.0348, -0.0045)**\\
    Body mass index (BMI)& 0.0319 (-0.1970,  0.2607)&-0.3033 (-0.4508, -0.1559)***&-0.4659 (-0.6132, -0.3185)***\\
    Physical domestic violence&-0.0784 (-0.1076, -0.0491)***&-0.0454 (-0.0687, -0.0222)***&-0.0382 (-0.0587, -0.0178)***\\
    Sexual domestic violence&-0.0114 (-0.0292,  0.0064)&-0.0213 (-0.0365, -0.0062)***&-0.0198 (-0.0342, -0.0054)***\\
    Home birth (3–12 months ago)&-0.1263 (-0.1593, -0.0934)***&-0.0951 (-0.1231, -0.0671)***&-0.0547 (-0.0805, -0.0288)***\\
    C-section (3–12 months ago)& 0.0370 ( 0.0166,  0.0575)***& 0.0357 ( 0.0098,  0.0617)***& 0.0549 ( 0.0233,  0.0866)***\\
&&&\\
\textbf{Panel B. Patrilocal coefficient with controls}&&&\\
    Consumes meat/egg/fish at least weekly&-0.016 (-0.044, 0.011)&-0.001 (-0.016, 0.015)&-0.002 (-0.018, 0.014)\\
    Consumes dairy daily&0.039 (0.007, 0.071)**&0.019 (0.003, 0.035)**&0.018 (0.002, 0.034)**\\
    No say in own healthcare&0.133 (0.103, 0.163)***&0.050 (0.020, 0.080)***&0.051 (0.021, 0.081)***\\
    No say in family visits&0.167 (0.138, 0.196)***&0.079 (0.049, 0.110)***&0.053 (0.022, 0.084)***\\
    Any anemia&-0.013 (-0.046, 0.021)&-0.010 (-0.027, 0.008)&0.007 (-0.011, 0.024)\\
    Body mass index (BMI)&0.090 (-0.134, 0.313)&-0.023 (-0.180, 0.134)&-0.142 (-0.297, 0.013)*\\
    Physical domestic violence&-0.037 (-0.067, -0.006)**&-0.018 (-0.044, 0.008)&-0.014 (-0.037, 0.008)\\
    Sexual domestic violence&-0.007 (-0.026, 0.012)&-0.011 (-0.025, 0.003)&-0.008 (-0.021, 0.004)\\
    Home birth (3–12 months ago)&-0.008 (-0.039, 0.022)&-0.012 (-0.040, 0.015)&0.001 (-0.025, 0.027)\\
    C-section (3–12 months ago)&-0.007 (-0.026, 0.013)&-0.014 (-0.041, 0.013)&0.009 (-0.023, 0.041)\\
\bottomrule
\end{tabular}

                
       
    \end{threeparttable}
\end{table}

\end{landscape}






% \begin{table}[H]
%     \centering
%     \begin{threeparttable}[t]
%         \caption{: Coefficient on patrilocal indicator from a regression using only state fixed effects}
%         \label{tab:unadjusted}

        
%         \setlength{\tabcolsep}{3pt}
%         \renewcommand{\arraystretch}{1.2}

%         Table 2. Patrilocal–Nuclear Gaps without controls by NFHS Round
\begin{tabular}{lccc}
\hline
Outcome & 2005-06 & 2015-16 & 2019-21 \\\\
\hline
Consumes meat/egg/fish at least weekly&-0.002&0.000&-0.006\\
Consumes dairy daily&0.121***&0.084***&0.082***\\
No say in own healthcare&0.155***&0.061***&0.065***\\
No say in family visits&0.204***&0.084***&0.064***\\
Any anemia&-0.041***&-0.027***&-0.024***\\
Body mass index (BMI)&0.000&-0.333***&-0.509***\\
Physical domestic violence&-0.078***&-0.045***&-0.038***\\
Sexual domestic violence&-0.011&-0.021***&-0.020***\\
Home birth (3–12 months ago)&-0.125***&-0.095***&-0.055***\\
C-section (3–12 months ago)&0.037***&0.036***&0.055***\\
\hline
\end{tabular}

                
%         \begin{tablenotes}
%             \item notes on sample for national vars
%             \item notes on sample for state vars
%             \item notes on sample for dv vars
%             \item notes on sample for last birth vars
%         \end{tablenotes}
%     \end{threeparttable}
% \end{table}



% \begin{table}[H]
%     \centering
%     \begin{threeparttable}[t]
%         \caption{: Coefficient on patrilocal indicator from a regression using controls}
%         \label{tab:unadjusted}

        
%         \setlength{\tabcolsep}{3pt}
%         \renewcommand{\arraystretch}{1.2}

%         \begin{tabular}{lccc}
\hline
Outcome & 2005-06 & 2015-16 & 2019-21 \\\\
\hline
Consumes meat/egg/fish at least weekly&-0.004&0.005&-0.001\\
Consumes dairy daily&0.064***&0.047***&0.044***\\
No say in own healthcare&0.133***&0.046***&0.053***\\
No say in family visits&0.166***&0.076***&0.054***\\
Any anemia&-0.026&-0.019**&-0.005\\
Body mass index (BMI)&0.285**&0.187**&0.077\\
Physical domestic violence&-0.044***&-0.023*&-0.018\\
Sexual domestic violence&-0.008&-0.013*&-0.011*\\
Home birth (3–12 months ago)&-0.053***&-0.047***&-0.017\\
C-section (3–12 months ago)&0.011&0.019&0.035**\\
\hline
\end{tabular}

                
%         \begin{tablenotes}
%             \item controls: parity(1,2,3,4+), age 5 year bins, education attainment, social group, whether distance to facility is a barrier in accessing healthcare
%             \item notes on sample for national vars
%             \item notes on sample for state vars
%             \item notes on sample for dv vars
%             \item notes on sample for last birth vars
%         \end{tablenotes}
%     \end{threeparttable}
% \end{table}





\begin{figure}[H]
    \centering
    \includegraphics[width=\textwidth]{figures/No say in own healthcare_pregnant.png}

    \parbox{1\linewidth}{\footnotesize Notes: NFHS-2 asks "who decides on obtaining health care". NFHS-3 asks "Final say on own health care". NFHS-4/5 asks "person who usually decides on respondent's healthcare"}
\end{figure}


% \begin{figure}[H]
%     \centering
%     \includegraphics[width=\textwidth]{figures/No say in large purchases_pregnant.png}
%     \parbox{1\linewidth}{\footnotesize Notes: NFHS-2 asks "Who decides to purchase jewelry". NFHS-3 asks "Final say on making large household purchases". NFHS-4/5 asks "Person who usually decides on large household purchases"}
% \end{figure}


% \begin{figure}[H]
%     \centering
%     \includegraphics[width=\textwidth]{figures/Experienced physical domestic violence.png}
%     \parbox{1\linewidth}{\footnotesize Notes: NFHS-2 asks "has been beaten since age 15, and husband has beaten respondent". NFHS-3/4/5 asks "spouse ever pushed/slapped/punched etc.}
% \end{figure}

% \begin{figure}[H]
%     \centering
%     \includegraphics[width=\textwidth]{figures/Consumes dairy daily_pregnant.png}
% \end{figure}



% \begin{figure}[H]
%     \centering
%     \includegraphics[width=\textwidth]{figures/hhstruc_bmi_preg.png}
% \end{figure}


% \begin{figure}[H]
%     \centering
%     \includegraphics[width=\textwidth]{figures/Wealth index z score.png}
% \end{figure}

% \begin{figure}[H]
%     \centering
%     \includegraphics[width=\textwidth]{figures/Delivered by Csection last 3-12 mo._nonpregnant.png}
% \end{figure}

% \begin{figure}[H]
%     \centering
%     \includegraphics[width=\textwidth]{figures/Last birth occurred at home within 3-12 mo._nonpregnant.png}
% \end{figure}

% \begin{figure}[H]
%     \centering
%     \includegraphics[width=\textwidth]{figures/region facility round2.png}
% \end{figure}

% \begin{figure}[H]
%     \centering
%     \includegraphics[width=\textwidth]{figures/region facility round3.png}
% \end{figure}

% \begin{figure}[H]
%     \centering
%     \includegraphics[width=\textwidth]{figures/region facility round4.png}
% \end{figure}

% \begin{figure}[H]
%     \centering
%     \includegraphics[width=\textwidth]{figures/region facility round5.png}
% \end{figure}








% \begin{figure}[H]
%     \centering
%     \includegraphics[width=\textwidth]{figures/hhstruc non pregnant.png}
% \end{figure}

% \begin{table}[H]
%     \centering
%     \setlength{\tabcolsep}{4pt} % shrink column padding
%     \footnotesize % shrink text
%     \caption{: Household structure 3+ mopreg women are observed in by subgroup}
%     \label{tab:sumstat}
%     \begin{adjustbox}{width=\textwidth}
%         \begin{tabular}{lcccc}
\toprule
Group & NFHS-2 & NFHS-3 & NFHS-4 & NFHS-5 \\\\
\midrule
\textbf{Nuclear Households}&&&&\\
India&29.4 (28.0, 30.9)&32.9 (31.1, 34.7)&26.1 (25.4, 26.9)&24.6 (23.9, 25.3)\\
Rural&29.6 (26.7, 32.6)&33.7 (30.8, 36.6)&28.1 (26.3, 29.9)&26.5 (24.8, 28.3)\\
Urban&29.4 (27.7, 31.0)&32.6 (30.5, 34.8)&25.4 (24.6, 26.1)&23.9 (23.1, 24.7)\\
Forward Caste&36.5 (31.5, 41.5)&40.2 (35.1, 45.2)&30.5 (28.6, 32.5)&29.7 (27.0, 32.4)\\
OBC&32.0 (28.7, 35.3)&38.8 (35.1, 42.6)&28.4 (26.7, 30.0)&25.9 (24.4, 27.4)\\
Dalit&26.8 (24.2, 29.4)&29.9 (27.0, 32.9)&23.7 (22.6, 24.8)&20.9 (19.8, 22.1)\\
Adivasi&25.0 (22.2, 27.7)&24.0 (20.7, 27.3)&20.2 (18.2, 22.2)&19.9 (17.8, 21.9)\\
Muslim&33.2 (29.3, 37.1)&38.2 (33.2, 43.2)&31.7 (29.8, 33.7)&32.0 (30.0, 34.1)\\
Sikh, Jain, Christian&25.8 (17.2, 34.4)&20.2 (11.9, 28.6)&18.5 (13.3, 23.6)&17.3 (11.1, 23.6)\\
.&&&&\\
\textbf{Sasural Households}&&&&\\
India&54.3 (52.8, 55.8)&51.7 (49.9, 53.5)&58.5 (57.6, 59.3)&59.9 (59.1, 60.8)\\
Rural&53.2 (50.1, 56.3)&51.8 (48.7, 54.9)&56.0 (53.9, 58.1)&57.8 (56.0, 59.7)\\
Urban&54.6 (52.9, 56.4)&51.7 (49.6, 53.9)&59.5 (58.6, 60.3)&60.7 (59.7, 61.6)\\
Forward Caste&51.2 (46.3, 56.1)&46.3 (41.1, 51.5)&55.9 (53.7, 58.1)&58.3 (55.6, 60.9)\\
OBC&54.4 (51.0, 57.8)&48.5 (44.8, 52.2)&57.1 (55.3, 58.9)&58.6 (56.9, 60.2)\\
Dalit&55.1 (52.3, 58.0)&52.3 (48.9, 55.6)&60.3 (59.1, 61.6)&62.5 (61.2, 63.8)\\
Adivasi&57.4 (54.3, 60.4)&60.7 (57.0, 64.4)&64.8 (62.6, 67.1)&64.6 (61.9, 67.2)\\
Muslim&49.1 (45.0, 53.2)&47.0 (42.5, 51.5)&51.6 (49.4, 53.8)&53.1 (51.0, 55.3)\\
Sikh, Jain, Christian&62.8 (52.2, 73.4)&64.6 (54.0, 75.1)&65.4 (59.0, 71.8)&62.4 (54.6, 70.1)\\
.&&&&\\
\textbf{Natal Households}&&&&\\
India&16.3 (15.1, 17.5)&15.4 (14.1, 16.6)&15.4 (14.8, 16.0)&15.5 (14.8, 16.1)\\
Rural&17.2 (14.7, 19.6)&14.5 (12.3, 16.7)&15.9 (14.5, 17.3)&15.6 (14.2, 17.0)\\
Urban&16.0 (14.7, 17.4)&15.7 (14.2, 17.1)&15.2 (14.5, 15.8)&15.4 (14.7, 16.1)\\
Forward Caste&12.3 (9.4, 15.1)&13.5 (10.1, 17.0)&13.5 (12.0, 15.1)&12.0 (10.5, 13.5)\\
OBC&13.6 (11.1, 16.1)&12.7 (10.3, 15.0)&14.5 (13.3, 15.8)&15.5 (14.3, 16.7)\\
Dalit&18.0 (15.6, 20.5)&17.8 (15.4, 20.1)&16.0 (15.0, 16.9)&16.5 (15.5, 17.6)\\
Adivasi&17.7 (15.3, 20.1)&15.2 (12.5, 18.0)&14.9 (13.2, 16.6)&15.5 (13.1, 18.0)\\
Muslim&17.7 (14.5, 20.9)&14.7 (11.5, 18.0)&16.7 (15.0, 18.4)&14.8 (13.2, 16.4)\\
Sikh, Jain, Christian&11.4 (5.2, 17.6)&15.2 (8.1, 22.4)&16.2 (11.4, 20.9)&20.3 (13.4, 27.2)\\
.&&&&\\
&&&&\\
\bottomrule
\end{tabular}

%     \end{adjustbox}
% \end{table}

% \begin{table}[!htbp]
%     \centering
%     \begin{threeparttable}[t]
%         \caption{:  Proportion of pregnant women who report having no say in own healthcare}
%         \label{tab:sumstats}

%         \scriptsize
%         \setlength{\tabcolsep}{2pt}
%         \renewcommand{\arraystretch}{1.2}

%         \begin{tabular}{l*{3}{>{\centering\arraybackslash}p{1.4cm}}}
\toprule
 & Nuclear & Patrilocal & Natal \\\\
\midrule
\textbf{NFHS-2 (1998–99)}&&&\\
\hspace*{2em}Adivasi&.57&.63&.58\\
\hspace*{2em}Dalit&.55&.56&.56\\
\hspace*{2em}OBC&.53&.57&.56\\
\hspace*{2em}Forward caste&.56&.56&.62\\
\hspace*{2em}Muslim&.55&.58&.60\\
&&&\\
\textbf{NFHS-3 (2005–06)}&&&\\
\hspace*{2em}Adivasi&.42&.58&.48\\
\hspace*{2em}Dalit&.33&.50&.51\\
\hspace*{2em}OBC&.35&.57&.53\\
\hspace*{2em}Forward caste&.31&.50&.49\\
\hspace*{2em}Muslim&.38&.55&.51\\
&&&\\
\textbf{NFHS-4 (2015–16)}&&&\\
\hspace*{2em}Adivasi&.25&.28&.24\\
\hspace*{2em}Dalit&.28&.33&.21\\
\hspace*{2em}OBC&.27&.35&.18\\
\hspace*{2em}Forward caste&.26&.33&.22\\
\hspace*{2em}Muslim&.23&.31&.18\\
&&&\\
\textbf{NFHS-5 (2019–21)}&&&\\
\hspace*{2em}Adivasi&.17&.21&.11\\
\hspace*{2em}Dalit&.20&.23&.20\\
\hspace*{2em}OBC&.20&.28&.17\\
\hspace*{2em}Forward caste&.16&.23&.17\\
\hspace*{2em}Muslim&.25&.31&.24\\
\bottomrule
\end{tabular}

                
%         \begin{tablenotes}
%             \item NFHS-2: who decides on obtaining healthcare
%             \item NFHS-3: final say on own healthcare
%             \item NFHS-4/5: person who usually decides on respondent's healthcare
%         \end{tablenotes}
%     \end{threeparttable}
% \end{table}


% \begin{figure}[H]
%     \centering
%     \includegraphics[width=\textwidth]{figures/No say in visiting natal family_pregnant.png}
%     \parbox{1\linewidth}{\footnotesize Notes: NFHS-2 asks "permission needed to visit relatives or friends. NFHS-3 asks "final say on visit to family or relatives". NFHS-4/5 asks "person who usually decides on visits to family or relatives}
% \end{figure}


% \begin{figure}[H]
%     \centering
%     \includegraphics[width=\textwidth]{figures/Consumes meat,egg,fish daily_pregnant.png}
% \end{figure}


% \begin{figure}[H]
%     \centering
%     \includegraphics[width=\textwidth]{figures/Any anemia (DHS cutoff for pregnancy)_pregnant.png}
% \end{figure}


% \begin{figure}[H]
%     \centering
%     \includegraphics[width=\textwidth]{figures/Experienced sexual domestic violence_pregnant.png}
% \end{figure}



\bibliographystyle{apalike}
\bibliography{references}
\end{document}
