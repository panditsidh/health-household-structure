\section{Results}

% \begin{figure}[H]
%     \centering
%     \includegraphics[width=\textwidth]{figures/hhstruc pregnant.png}
%     \caption{: trends in household structure over time}
% \end{figure}


% \section{Autonomy}

% \begin{figure}[H]
%     \centering
%     \includegraphics[width=\textwidth]{figures3_4_5/autonomy/Getting permission is a problem for healthcare access_pregnant.png}
%     \caption{: Change in getting permission to access healthcare}
% \end{figure}

% \begin{figure}[H]
%     \centering
%     \includegraphics[width=\textwidth]{figures3_4_5/autonomy/Money is a problem for healthcare access_pregnant.png}
%     \caption{:Money as a barrier to healthcare}
% \end{figure}

% \begin{figure}[H]
%     \centering
%     \includegraphics[width=\textwidth]{figures3_4_5/autonomy/No say in own healthcare_pregnant.png}
%     \caption{: Having say in healthcare}
% \end{figure}






% \section{Nutrition}

% \begin{figure}[H]
%     \centering
%     \includegraphics[width=\textwidth]{figures3_4_5/nutrition/dairy_daily_pregnant.png}
%     \caption{:Daily dairy consumption}
% \end{figure}

% \begin{figure}[H]
%     \centering
%     \includegraphics[width=\textwidth]{figures3_4_5/nutrition/meat_egg_fish_daily_pregnant.png}
%     \caption{:Consumption of egg, meat and fish}
% \end{figure}


% \begin{figure}[H]
%     \centering
%     \includegraphics[width=\textwidth]{figures3_4_5/nutrition/COEFS bmi .png}
%     \caption{:BMI}
% \end{figure}

% \begin{figure}[H]
%     \centering
%     \includegraphics[width=\textwidth]{figures3_4_5/nutrition/anemic_pregnant.png}
%     \caption{:anemic}
% \end{figure}




% \section{Healthcare at birth}

% \begin{figure}[H]
%     \centering
%     \includegraphics[width=\textwidth]{figures3_4_5/healthcare at birth/COEFS home birth .png}
%     \caption{:Birth at Home}
% \end{figure}

% \begin{figure}[H]
%     \centering
%     \includegraphics[width=\textwidth]{figures3_4_5/healthcare at birth/COEFS private birth .png}
%     \caption{:Birth at private facility}
% \end{figure}

% \begin{figure}[H]
%     \centering
%     \includegraphics[width=\textwidth]{figures3_4_5/healthcare at birth/COEFS public birth .png}
%     \caption{:Birth at public facility}
% \end{figure}


% \begin{itemize}

%     \item \textbf{Figure 1: Distribution of household structure across all NFHS rounds.}  
%     This establishes the broader demographic context: nuclear, natal, and patrilocal households across five survey rounds.

%     \item \textbf{Table 1: Household structure by subgroup (Allendorf-style).}  
%     We show the percent of pregnant women living in patrilocal households by subgroup (caste, religion, parity, education, region) for NFHS-3, NFHS-4, and NFHS-5.  
%     This demonstrates that the increase in patrilocal living is \emph{broad-based} across nearly all groups.  
%     (Here we can briefly mention the 30\% parity decomposition result to show that part of the change is compositional.)

%     \item \textbf{Figure 2: Trends in key outcomes by household structure.}  
%     Four panels: no say in healthcare, no say due to money, dairy daily consumption, and BMI.  
%     These descriptive figures show:
%     \begin{itemize}
%         \item Women's autonomy is consistently \emph{lower} in patrilocal households, though gaps narrow over time.
%         \item Money as a barrier to healthcare is most common in nuclear households.
%         \item Dairy consumption is much higher in patrilocal households.
%         \item BMI is highest among nuclear women despite lower dairy intake, suggesting complex pathways.
%     \end{itemize}
%     These patterns show that autonomy disadvantages coexist with material advantages in patrilocal households, motivating a regression-based approach.

%     \item \textbf{Table 2: Descriptive statistics in NFHS-5 (Allendorf-style).}  
%     For each outcome, we report means for: all women, patrilocal households, and nuclear households, with significance of differences.  
%     This shows the cross-sectional inequalities by household structure in the most recent round.

%     \item \textbf{Tables 3 and 4: Patrilocal--nuclear regression gaps (unadjusted and adjusted).}  
%     Columns are NFHS-3, NFHS-4, and NFHS-5; rows are outcomes.  
%     Table 3 includes only state fixed effects.  
%     Table 4 adds controls (age, education, caste/religion, urban residence, and gestational duration where relevant).  
%     These tables show:

%     \begin{itemize}

%         \item \textbf{Convergence over time:}  
%         Coefficients shrink across rounds for almost all outcomes, indicating convergence in norms and living conditions across household structures.

%         \item \textbf{Persistent autonomy disadvantage:}  
%         Patrilocal women consistently have less say in their own healthcare and mobility.  
%         Gaps narrow but remain large and statistically significant even after full controls, suggesting a structural feature of patrilocal living.

%         \item \textbf{Stable patrilocal advantage in dairy consumption:}  
%         Large, positive, and significant in all rounds; remains robust with controls.  
%         Indicates persistent material advantages in patrilocal households.

%         \item \textbf{Domestic violence consistently lower in patrilocal households:}  
%         Physical violence is significantly lower for patrilocal women in all rounds; sexual violence is modestly lower.  m

%         \item \textbf{Home birth gap persists but shrinks sharply:}  
%         Patrilocal women are less likely to deliver at home; gaps decline substantially over time and with controls.  
%         Suggests that patrilocal households facilitate institutional delivery (wealth, transport, social networks).

%         \item \textbf{C-section rates higher in patrilocal households:}  
%         Coefficients are consistently positive.  
%         Even after controls, patrilocal women are more likely to have C-sections, consistent with higher use of private facilities.

%         \item \textbf{BMI reversal after controls:}  
%         Unadjusted: patrilocal women are thinner in later rounds.  
%         Adjusted: patrilocal women have slightly \emph{higher} BMI.  
%         Indicates that raw BMI gaps are compositional (education, wealth, caste).

%         \item \textbf{Anemia differences disappear after controls:}  
%         Unadjusted gaps favor patrilocal women, but become small or insignificant after controls.  
%         Household structure has little independent association with anemia.

%     \end{itemize}

%     TODO TABLE 4: neonatal survival regressions

%     Together, these patterns suggest that increasing patrilocality does \emph{not} uniformly worsen women's health.  Patrilocal living is associated with lower autonomy but better material conditions. Over time, as nuclear and patrilocal households converge in wealth and norms, gaps shrink across nearly all outcomes.

    

% \end{itemize}





% \section{Appendix}

    
% \begin{table}[H]
%     \centering
%     \begin{threeparttable}[t]
%         \caption{: Descriptive stats}
%         \label{tab:unadjusted}

%         \scriptsize
%         \setlength{\tabcolsep}{3pt}
%         \renewcommand{\arraystretch}{1}

%         \begin{tabular}{lcccc}
\toprule
Outcome & All & Nuclear & Patrilocal & Sig. \\\\
\midrule
&&&&\\
\textbf{Health}&&&&\\
    BMI (residualized)&0.038&0.406&-0.188&***\\
    Underweight&0.117&0.113&0.119&\\
    Any anemia&0.580&0.595&0.570&***\\
&&&&\\
\textbf{Diet}&&&&\\
    Consumes meat/egg/fish weekly&0.456&0.499&0.429&***\\
    Consumes dairy daily&0.522&0.465&0.557&***\\
&&&&\\
\textbf{Healthcare access}&&&&\\
    Distance is a problem (facility)&0.607&0.624&0.596&***\\
    Money is a problem (healthcare)&0.534&0.562&0.517&***\\
    4+ ANC visits&0.538&0.487&0.566&***\\
&&&&\\
\textbf{Region}&&&&\\
    North region&0.117&0.085&0.136&***\\
    Central region&0.084&0.077&0.088&***\\
    East region&0.137&0.160&0.123&***\\
    West region&0.143&0.126&0.154&***\\
    South region&0.170&0.193&0.155&***\\
    Northeast region&0.037&0.045&0.032&***\\
&&&&\\
\textbf{Social group}&&&&\\
    Forward caste&0.099&0.104&0.095&*\\
    OBC&0.225&0.233&0.220&*\\
    Dalit&0.357&0.327&0.375&***\\
    Adivasi&0.157&0.147&0.162&**\\
    Muslim&0.150&0.178&0.133&***\\
    Sikh/Jain/Christian&0.012&0.010&0.013&**\\
&&&&\\
\textbf{Education}&&&&\\
    No education&0.210&0.320&0.142&***\\
    Primary education&0.119&0.158&0.095&***\\
    Secondary education&0.518&0.427&0.573&***\\
    Higher education&0.154&0.095&0.190&***\\
&&&&\\
\textbf{Wealth}&&&&\\
    Poorest quintile&0.216&0.294&0.168&***\\
    Poorer quintile&0.213&0.224&0.207&**\\
    Middle quintile&0.199&0.183&0.208&***\\
    Richer quintile&0.196&0.163&0.216&***\\
    Richest quintile&0.176&0.136&0.201&***\\
&&&&\\
\textbf{Assets}&&&&\\
    Owns fridge&0.378&0.289&0.433&***\\
    Owns television&0.663&0.576&0.717&***\\
    Owns motorcycle&0.568&0.425&0.656&***\\
&&&&\\
\textbf{Parity}&&&&\\
    Parity: 1 child&0.281&0.215&0.321&***\\
    Parity: 2 children&0.231&0.286&0.197&***\\
    Parity: 3 children&0.119&0.183&0.080&***\\
    Parity: 4+ children&0.369&0.316&0.402&***\\
&&&&\\
\textbf{Decision making}&&&&\\
    No say in own healthcare&0.218&0.176&0.246&***\\
    Permission is a problem&0.381&0.366&0.392&***\\
    No say in family visits&0.219&0.175&0.248&***\\
    No say in purchases&0.243&0.183&0.284&***\\
&&&&\\
\textbf{Domestic violence}&&&&\\
    Physical domestic violence&0.069&0.089&0.055&***\\
    Sexual domestic violence&0.026&0.036&0.018&**\\
\bottomrule
\end{tabular}

                
%         \begin{tablenotes}
%             \item sample now is only patrilocal and nuclear
%         \end{tablenotes}
%     \end{threeparttable}
% \end{table}



% \begin{landscape}


% \begin{table}[H]
%     \centering
%     \begin{threeparttable}[t]
%         \caption{: patrilocal coeff}
%         \label{tab:unadjusted}

%         \scriptsize
%         \setlength{\tabcolsep}{3pt}
%         \renewcommand{\arraystretch}{1.7}

%         \begin{tabular}{lccc}
\toprule
Outcome & 2005-06 & 2015-16 & 2019-21 \\\\
\midrule
\textbf{Panel A. Patrilocal coefficient without controls}&&&\\
    Consumes meat/egg/fish at least weekly&-0.0016 (-0.0271,  0.0238)& 0.0003 (-0.0143,  0.0149)&-0.0063 (-0.0209,  0.0083)\\
    Consumes dairy daily& 0.1215 ( 0.0898,  0.1532)***& 0.0845 ( 0.0697,  0.0993)***& 0.0815 ( 0.0668,  0.0962)***\\
    No say in own healthcare& 0.1546 ( 0.1268,  0.1825)***& 0.0615 ( 0.0340,  0.0890)***& 0.0648 ( 0.0375,  0.0922)***\\
    No say in family visits& 0.2041 ( 0.1769,  0.2312)***& 0.0845 ( 0.0565,  0.1125)***& 0.0641 ( 0.0368,  0.0914)***\\
    Any anemia&-0.0365 (-0.0674, -0.0056)**&-0.0231 (-0.0387, -0.0075)***&-0.0197 (-0.0348, -0.0045)**\\
    Body mass index (BMI)& 0.0319 (-0.1970,  0.2607)&-0.3033 (-0.4508, -0.1559)***&-0.4659 (-0.6132, -0.3185)***\\
    Physical domestic violence&-0.0784 (-0.1076, -0.0491)***&-0.0454 (-0.0687, -0.0222)***&-0.0382 (-0.0587, -0.0178)***\\
    Sexual domestic violence&-0.0114 (-0.0292,  0.0064)&-0.0213 (-0.0365, -0.0062)***&-0.0198 (-0.0342, -0.0054)***\\
    Home birth (3–12 months ago)&-0.1263 (-0.1593, -0.0934)***&-0.0951 (-0.1231, -0.0671)***&-0.0547 (-0.0805, -0.0288)***\\
    C-section (3–12 months ago)& 0.0370 ( 0.0166,  0.0575)***& 0.0357 ( 0.0098,  0.0617)***& 0.0549 ( 0.0233,  0.0866)***\\
&&&\\
\textbf{Panel B. Patrilocal coefficient with controls}&&&\\
    Consumes meat/egg/fish at least weekly&-0.016 (-0.044, 0.011)&-0.001 (-0.016, 0.015)&-0.002 (-0.018, 0.014)\\
    Consumes dairy daily&0.039 (0.007, 0.071)**&0.019 (0.003, 0.035)**&0.018 (0.002, 0.034)**\\
    No say in own healthcare&0.133 (0.103, 0.163)***&0.050 (0.020, 0.080)***&0.051 (0.021, 0.081)***\\
    No say in family visits&0.167 (0.138, 0.196)***&0.079 (0.049, 0.110)***&0.053 (0.022, 0.084)***\\
    Any anemia&-0.013 (-0.046, 0.021)&-0.010 (-0.027, 0.008)&0.007 (-0.011, 0.024)\\
    Body mass index (BMI)&0.090 (-0.134, 0.313)&-0.023 (-0.180, 0.134)&-0.142 (-0.297, 0.013)*\\
    Physical domestic violence&-0.037 (-0.067, -0.006)**&-0.018 (-0.044, 0.008)&-0.014 (-0.037, 0.008)\\
    Sexual domestic violence&-0.007 (-0.026, 0.012)&-0.011 (-0.025, 0.003)&-0.008 (-0.021, 0.004)\\
    Home birth (3–12 months ago)&-0.008 (-0.039, 0.022)&-0.012 (-0.040, 0.015)&0.001 (-0.025, 0.027)\\
    C-section (3–12 months ago)&-0.007 (-0.026, 0.013)&-0.014 (-0.041, 0.013)&0.009 (-0.023, 0.041)\\
\bottomrule
\end{tabular}

                
       
%     \end{threeparttable}
% \end{table}

% \end{landscape}






% % \begin{table}[H]
% %     \centering
% %     \begin{threeparttable}[t]
% %         \caption{: Coefficient on patrilocal indicator from a regression using only state fixed effects}
% %         \label{tab:unadjusted}

        
% %         \setlength{\tabcolsep}{3pt}
% %         \renewcommand{\arraystretch}{1.2}

% %         Table 2. Patrilocal–Nuclear Gaps without controls by NFHS Round
\begin{tabular}{lccc}
\hline
Outcome & 2005-06 & 2015-16 & 2019-21 \\\\
\hline
Consumes meat/egg/fish at least weekly&-0.002&0.000&-0.006\\
Consumes dairy daily&0.121***&0.084***&0.082***\\
No say in own healthcare&0.155***&0.061***&0.065***\\
No say in family visits&0.204***&0.084***&0.064***\\
Any anemia&-0.041***&-0.027***&-0.024***\\
Body mass index (BMI)&0.000&-0.333***&-0.509***\\
Physical domestic violence&-0.078***&-0.045***&-0.038***\\
Sexual domestic violence&-0.011&-0.021***&-0.020***\\
Home birth (3–12 months ago)&-0.125***&-0.095***&-0.055***\\
C-section (3–12 months ago)&0.037***&0.036***&0.055***\\
\hline
\end{tabular}

                
% %         \begin{tablenotes}
% %             \item notes on sample for national vars
% %             \item notes on sample for state vars
% %             \item notes on sample for dv vars
% %             \item notes on sample for last birth vars
% %         \end{tablenotes}
% %     \end{threeparttable}
% % \end{table}



% % \begin{table}[H]
% %     \centering
% %     \begin{threeparttable}[t]
% %         \caption{: Coefficient on patrilocal indicator from a regression using controls}
% %         \label{tab:unadjusted}

        
% %         \setlength{\tabcolsep}{3pt}
% %         \renewcommand{\arraystretch}{1.2}

% %         \begin{tabular}{lccc}
\hline
Outcome & 2005-06 & 2015-16 & 2019-21 \\\\
\hline
Consumes meat/egg/fish at least weekly&-0.004&0.005&-0.001\\
Consumes dairy daily&0.064***&0.047***&0.044***\\
No say in own healthcare&0.133***&0.046***&0.053***\\
No say in family visits&0.166***&0.076***&0.054***\\
Any anemia&-0.026&-0.019**&-0.005\\
Body mass index (BMI)&0.285**&0.187**&0.077\\
Physical domestic violence&-0.044***&-0.023*&-0.018\\
Sexual domestic violence&-0.008&-0.013*&-0.011*\\
Home birth (3–12 months ago)&-0.053***&-0.047***&-0.017\\
C-section (3–12 months ago)&0.011&0.019&0.035**\\
\hline
\end{tabular}

                
% %         \begin{tablenotes}
% %             \item controls: parity(1,2,3,4+), age 5 year bins, education attainment, social group, whether distance to facility is a barrier in accessing healthcare
% %             \item notes on sample for national vars
% %             \item notes on sample for state vars
% %             \item notes on sample for dv vars
% %             \item notes on sample for last birth vars
% %         \end{tablenotes}
% %     \end{threeparttable}
% % \end{table}





% \begin{figure}[H]
%     \centering
%     \includegraphics[width=\textwidth]{figures/No say in own healthcare_pregnant.png}

%     \parbox{1\linewidth}{\footnotesize Notes: NFHS-2 asks "who decides on obtaining health care". NFHS-3 asks "Final say on own health care". NFHS-4/5 asks "person who usually decides on respondent's healthcare"}
% \end{figure}


% % \begin{figure}[H]
% %     \centering
% %     \includegraphics[width=\textwidth]{figures/No say in large purchases_pregnant.png}
% %     \parbox{1\linewidth}{\footnotesize Notes: NFHS-2 asks "Who decides to purchase jewelry". NFHS-3 asks "Final say on making large household purchases". NFHS-4/5 asks "Person who usually decides on large household purchases"}
% % \end{figure}


% % \begin{figure}[H]
% %     \centering
% %     \includegraphics[width=\textwidth]{figures/Experienced physical domestic violence.png}
% %     \parbox{1\linewidth}{\footnotesize Notes: NFHS-2 asks "has been beaten since age 15, and husband has beaten respondent". NFHS-3/4/5 asks "spouse ever pushed/slapped/punched etc.}
% % \end{figure}

% % \begin{figure}[H]
% %     \centering
% %     \includegraphics[width=\textwidth]{figures/Consumes dairy daily_pregnant.png}
% % \end{figure}



% % \begin{figure}[H]
% %     \centering
% %     \includegraphics[width=\textwidth]{figures/hhstruc_bmi_preg.png}
% % \end{figure}


% % \begin{figure}[H]
% %     \centering
% %     \includegraphics[width=\textwidth]{figures/Wealth index z score.png}
% % \end{figure}

% % \begin{figure}[H]
% %     \centering
% %     \includegraphics[width=\textwidth]{figures/Delivered by Csection last 3-12 mo._nonpregnant.png}
% % \end{figure}

% % \begin{figure}[H]
% %     \centering
% %     \includegraphics[width=\textwidth]{figures/Last birth occurred at home within 3-12 mo._nonpregnant.png}
% % \end{figure}

% % \begin{figure}[H]
% %     \centering
% %     \includegraphics[width=\textwidth]{figures/region facility round2.png}
% % \end{figure}

% % \begin{figure}[H]
% %     \centering
% %     \includegraphics[width=\textwidth]{figures/region facility round3.png}
% % \end{figure}

% % \begin{figure}[H]
% %     \centering
% %     \includegraphics[width=\textwidth]{figures/region facility round4.png}
% % \end{figure}

% % \begin{figure}[H]
% %     \centering
% %     \includegraphics[width=\textwidth]{figures/region facility round5.png}
% % \end{figure}








% % \begin{figure}[H]
% %     \centering
% %     \includegraphics[width=\textwidth]{figures/hhstruc non pregnant.png}
% % \end{figure}

% % \begin{table}[H]
% %     \centering
% %     \setlength{\tabcolsep}{4pt} % shrink column padding
% %     \footnotesize % shrink text
% %     \caption{: Household structure 3+ mopreg women are observed in by subgroup}
% %     \label{tab:sumstat}
% %     \begin{adjustbox}{width=\textwidth}
% %         \begin{tabular}{lcccc}
\toprule
Group & NFHS-2 & NFHS-3 & NFHS-4 & NFHS-5 \\\\
\midrule
\textbf{Nuclear Households}&&&&\\
India&29.4 (28.0, 30.9)&32.9 (31.1, 34.7)&26.1 (25.4, 26.9)&24.6 (23.9, 25.3)\\
Rural&29.6 (26.7, 32.6)&33.7 (30.8, 36.6)&28.1 (26.3, 29.9)&26.5 (24.8, 28.3)\\
Urban&29.4 (27.7, 31.0)&32.6 (30.5, 34.8)&25.4 (24.6, 26.1)&23.9 (23.1, 24.7)\\
Forward Caste&36.5 (31.5, 41.5)&40.2 (35.1, 45.2)&30.5 (28.6, 32.5)&29.7 (27.0, 32.4)\\
OBC&32.0 (28.7, 35.3)&38.8 (35.1, 42.6)&28.4 (26.7, 30.0)&25.9 (24.4, 27.4)\\
Dalit&26.8 (24.2, 29.4)&29.9 (27.0, 32.9)&23.7 (22.6, 24.8)&20.9 (19.8, 22.1)\\
Adivasi&25.0 (22.2, 27.7)&24.0 (20.7, 27.3)&20.2 (18.2, 22.2)&19.9 (17.8, 21.9)\\
Muslim&33.2 (29.3, 37.1)&38.2 (33.2, 43.2)&31.7 (29.8, 33.7)&32.0 (30.0, 34.1)\\
Sikh, Jain, Christian&25.8 (17.2, 34.4)&20.2 (11.9, 28.6)&18.5 (13.3, 23.6)&17.3 (11.1, 23.6)\\
.&&&&\\
\textbf{Sasural Households}&&&&\\
India&54.3 (52.8, 55.8)&51.7 (49.9, 53.5)&58.5 (57.6, 59.3)&59.9 (59.1, 60.8)\\
Rural&53.2 (50.1, 56.3)&51.8 (48.7, 54.9)&56.0 (53.9, 58.1)&57.8 (56.0, 59.7)\\
Urban&54.6 (52.9, 56.4)&51.7 (49.6, 53.9)&59.5 (58.6, 60.3)&60.7 (59.7, 61.6)\\
Forward Caste&51.2 (46.3, 56.1)&46.3 (41.1, 51.5)&55.9 (53.7, 58.1)&58.3 (55.6, 60.9)\\
OBC&54.4 (51.0, 57.8)&48.5 (44.8, 52.2)&57.1 (55.3, 58.9)&58.6 (56.9, 60.2)\\
Dalit&55.1 (52.3, 58.0)&52.3 (48.9, 55.6)&60.3 (59.1, 61.6)&62.5 (61.2, 63.8)\\
Adivasi&57.4 (54.3, 60.4)&60.7 (57.0, 64.4)&64.8 (62.6, 67.1)&64.6 (61.9, 67.2)\\
Muslim&49.1 (45.0, 53.2)&47.0 (42.5, 51.5)&51.6 (49.4, 53.8)&53.1 (51.0, 55.3)\\
Sikh, Jain, Christian&62.8 (52.2, 73.4)&64.6 (54.0, 75.1)&65.4 (59.0, 71.8)&62.4 (54.6, 70.1)\\
.&&&&\\
\textbf{Natal Households}&&&&\\
India&16.3 (15.1, 17.5)&15.4 (14.1, 16.6)&15.4 (14.8, 16.0)&15.5 (14.8, 16.1)\\
Rural&17.2 (14.7, 19.6)&14.5 (12.3, 16.7)&15.9 (14.5, 17.3)&15.6 (14.2, 17.0)\\
Urban&16.0 (14.7, 17.4)&15.7 (14.2, 17.1)&15.2 (14.5, 15.8)&15.4 (14.7, 16.1)\\
Forward Caste&12.3 (9.4, 15.1)&13.5 (10.1, 17.0)&13.5 (12.0, 15.1)&12.0 (10.5, 13.5)\\
OBC&13.6 (11.1, 16.1)&12.7 (10.3, 15.0)&14.5 (13.3, 15.8)&15.5 (14.3, 16.7)\\
Dalit&18.0 (15.6, 20.5)&17.8 (15.4, 20.1)&16.0 (15.0, 16.9)&16.5 (15.5, 17.6)\\
Adivasi&17.7 (15.3, 20.1)&15.2 (12.5, 18.0)&14.9 (13.2, 16.6)&15.5 (13.1, 18.0)\\
Muslim&17.7 (14.5, 20.9)&14.7 (11.5, 18.0)&16.7 (15.0, 18.4)&14.8 (13.2, 16.4)\\
Sikh, Jain, Christian&11.4 (5.2, 17.6)&15.2 (8.1, 22.4)&16.2 (11.4, 20.9)&20.3 (13.4, 27.2)\\
.&&&&\\
&&&&\\
\bottomrule
\end{tabular}

% %     \end{adjustbox}
% % \end{table}

% % \begin{table}[!htbp]
% %     \centering
% %     \begin{threeparttable}[t]
% %         \caption{:  Proportion of pregnant women who report having no say in own healthcare}
% %         \label{tab:sumstats}

% %         \scriptsize
% %         \setlength{\tabcolsep}{2pt}
% %         \renewcommand{\arraystretch}{1.2}

% %         \begin{tabular}{l*{3}{>{\centering\arraybackslash}p{1.4cm}}}
\toprule
 & Nuclear & Patrilocal & Natal \\\\
\midrule
\textbf{NFHS-2 (1998–99)}&&&\\
\hspace*{2em}Adivasi&.57&.63&.58\\
\hspace*{2em}Dalit&.55&.56&.56\\
\hspace*{2em}OBC&.53&.57&.56\\
\hspace*{2em}Forward caste&.56&.56&.62\\
\hspace*{2em}Muslim&.55&.58&.60\\
&&&\\
\textbf{NFHS-3 (2005–06)}&&&\\
\hspace*{2em}Adivasi&.42&.58&.48\\
\hspace*{2em}Dalit&.33&.50&.51\\
\hspace*{2em}OBC&.35&.57&.53\\
\hspace*{2em}Forward caste&.31&.50&.49\\
\hspace*{2em}Muslim&.38&.55&.51\\
&&&\\
\textbf{NFHS-4 (2015–16)}&&&\\
\hspace*{2em}Adivasi&.25&.28&.24\\
\hspace*{2em}Dalit&.28&.33&.21\\
\hspace*{2em}OBC&.27&.35&.18\\
\hspace*{2em}Forward caste&.26&.33&.22\\
\hspace*{2em}Muslim&.23&.31&.18\\
&&&\\
\textbf{NFHS-5 (2019–21)}&&&\\
\hspace*{2em}Adivasi&.17&.21&.11\\
\hspace*{2em}Dalit&.20&.23&.20\\
\hspace*{2em}OBC&.20&.28&.17\\
\hspace*{2em}Forward caste&.16&.23&.17\\
\hspace*{2em}Muslim&.25&.31&.24\\
\bottomrule
\end{tabular}

                
% %         \begin{tablenotes}
% %             \item NFHS-2: who decides on obtaining healthcare
% %             \item NFHS-3: final say on own healthcare
% %             \item NFHS-4/5: person who usually decides on respondent's healthcare
% %         \end{tablenotes}
% %     \end{threeparttable}
% % \end{table}


% % \begin{figure}[H]
% %     \centering
% %     \includegraphics[width=\textwidth]{figures/No say in visiting natal family_pregnant.png}
% %     \parbox{1\linewidth}{\footnotesize Notes: NFHS-2 asks "permission needed to visit relatives or friends. NFHS-3 asks "final say on visit to family or relatives". NFHS-4/5 asks "person who usually decides on visits to family or relatives}
% % \end{figure}


% % \begin{figure}[H]
% %     \centering
% %     \includegraphics[width=\textwidth]{figures/Consumes meat,egg,fish daily_pregnant.png}
% % \end{figure}


% % \begin{figure}[H]
% %     \centering
% %     \includegraphics[width=\textwidth]{figures/Any anemia (DHS cutoff for pregnancy)_pregnant.png}
% % \end{figure}


% % \begin{figure}[H]
% %     \centering
% %     \includegraphics[width=\textwidth]{figures/Experienced sexual domestic violence_pregnant.png}
% % \end{figure}
