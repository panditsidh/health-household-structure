\documentclass[12pt]{article}
\usepackage{graphicx} %package to manage images
\graphicspath{ {./figures/} }
\usepackage{caption}
\usepackage[font=scriptsize]{subcaption}
\captionsetup[figure]{labelsep=none}
\captionsetup[table]{labelsep=none}
\usepackage{bbm}
\usepackage{amsmath}
\usepackage{import}
\usepackage{array}
\usepackage{booktabs}     % no extra vertical space between paragraphs
\usepackage{indentfirst}
\usepackage{afterpage}
\usepackage{floatrow}
\usepackage{pdflscape}
\usepackage{soul}
\usepackage{float}
\usepackage{adjustbox}
\usepackage{longtable}
\usepackage{caption}
\usepackage{setspace}
\usepackage{afterpage}
\usepackage[margin=1in]{geometry}
\usepackage[round]{natbib}
\setcitestyle{authoryear}
\usepackage{hyperref}
\usepackage{titlesec}
\usepackage{threeparttable}
\usepackage{setspace}
\usepackage{float}
\floatstyle{plaintop}
\restylefloat{table}
\usepackage{rotating}
\usepackage{authblk}
\usepackage{makecell}


% Optional styling for nicer author/affil typography
\renewcommand\Authfont{\normalsize\bfseries}
\renewcommand\Affilfont{\small\normalfont}
\setlength{\affilsep}{0.3em} % space between author and affiliation

\setlength{\parindent}{0pt}   % standard indent (~0.25 inch)
\setlength{\parskip}{1em}       % no extra vertical space between paragraphs

\titleformat{\section}
  {\normalfont\bfseries\fontsize{14}{16}\selectfont}  % Bold 14pt
  {}{0pt}{}  % No numbering

\titleformat{\subsection}
  {\normalfont\bfseries\fontsize{12}{14}\selectfont}  % Bold 12pt
  {}{0pt}{}  % No numbering

\titleformat{\subsubsection}
  {\normalfont\itshape\fontsize{12}{14}\selectfont}  % Italic 12pt
  {}{0pt}{}  % No numbering
  
\begin{document}
\title{Consequence of increasing joint family residence on pregnant women's autonomy and healthcare in India: A Research Note}
\date{\today}
\maketitle

\section{Introduction}

Health during pregnancy matters for both women and children. The ability to make decisions about one’s own healthcare may have positive consequences for health and is good in itself. In Indian context historically women’s decision-making power in their marital household have been constrained. In these households married women are subject to more scrutiny and are expected to seek permission from senior household members before accessing healthcare or other resources.  

Within patrilocal joint households newly married women occupy a particularly disadvantaged position in terms of gender and generational hierarchies. They are located in the bottom of household and power structures (Dasgupta,1999). This further puts them in a vulnerable position when they want to take any decision pertaining to their own or their children’s health as they might have to go through different senior members of the household before any action is taken. 
Further evidence suggests that their social status has an effect not only in their own healthcare access but also with their children. Coffey et.al(2022) finds autonomy within the household differs by social ranking which is defined the woman’s husband’s birth order. In this context children of lower ranking mothers are more likely to die compared to their counterparts. 

In this study we focus on two specific sample of women that is women who are currently pregnant and women who gave birth in the last 3-12 months. Examining these two groups allows us to distinguish between women's autonomy over healthcare decision making and their realized access to healthcare services. Thus, we address the following research questions, How has the household structure in which pregnant women in India live changed since 2005?  What are the consequences of those changes for their autonomy and health care use?

\subsection{Research on household structure in India}



\subsection{Household structure and women's autonomy}

As Das Gupta (1999) observed in her work in a patrilineal joint family system women are structurally more disadvataged. As being in the bottom of the both generational and gender hierarchies’ young brides have layers of people above them for decision making and it’s just not men but also every woman who is senior to her. Hence women are mostly powerless in these household structures atleast in the early phases of their marriage which is also their peak childbearing years. 

\subsection{Healthcare during pregnancy}



\section{Data and methodology}

\section{Results}

Over the last three NFHS rounds that is from 2005 till 2021 we observe more pregnant women living in their patrilocal joint households compared to nuclear household structure. This goes against the believe of modernization theory which indicates a move towards nuclearization. We see that there is a increase in patrilocal joint household living from 47 percentage points to 53 percentage points as well as decline in nuclear household living from 38 pp to 32 pp. 

\begin{figure}[H]
    \centering
    \includegraphics[width=\textwidth] 
    {figures/figure 1 change in hhstructure.png}
    \caption{: 1 panel figure showing change showing household structure}
    \label{fig:ihds_eat_last}
    \parbox{1\linewidth}{\footnotesize figure notes go here}
\end{figure}

As we investigate what does this mean for pregnant women and their autonomy as well as healthcare access we observe that pregnant women in joint patrilocal households consistently has reported having lower say in their healthcare as well as visiting their families. The difference in having no say in their health care and visiting their families have gone down than before but the gap still exists. 
Looking at the wealth measures we see joint households are in on average better than nuclear households. In terms of wealth measure, we look into finished floor, electricity, owns radio, owns TV, owns refrigerator, owns bicycle, owns car, uses toilets/ latrines, owns land.
Thus when looking at women who gave birth 3-12 months before the survey , we observe among them women in joint patrilocal houses reported to have given birth in health facility as well as going to 4 + antenatal visits. 
Thus indicating though there is lower autonomy in patrilocal household structures women have better healthcare access. This can be an indication of wealth being a mediator when it comes to access to better healthcare services. 
There are consistent gaps in having “no say in own health care” and “no say in visits to family and friends” that suggest that pregnant women in joint households have less autonomy than those in nuclear households.  Yet, pregnant women in joint households enjoy apparent advantages in healthcare: they are less likely to give birth at home and more likely to have the WHO-recommended 4 prenatal visits.


\begin{landscape}
    

\begin{figure}[H]
    \centering
    \includegraphics[width=\textwidth]
    {figures/figure 2 four panel.pdf}
    \caption{: 4 panel figure showing changes over time in say in own healthcare, say in visits to family, friends, facility birth, and 4+ ANC visits}
    \label{fig:ihds_eat_last}
    \parbox{1\linewidth}{\footnotesize figure notes go here}
\end{figure}

\end{landscape}

In NFHS 3 women living in joint households had a 18.4 percentage point higher probability of reporting no say in their own health compared to women in nuclear households. The association has increased slightly when adjusted for wealth. Whereas, in NFHS 5 we see that the difference have gone down significantly where only 6pp women living in the joint patrilocal extended households claim to not have any say in their own healthcare. We don’t see much change when control for wealth in this case. 
Similarly for no say in visit to family and friends currently pregnant women living in joint patrilocal households were 24 pp more likely to not have say in visiting friends and family compared to their counterparts in nuclear households in 2005 the gap has decreased now to 6 pp but the fact that pregnant women in joint patrilocal households have less say has remained consistent. 
When it comes to healthcare, 
we observe that controlling for wealth has a significant effect, we observe that in NFHS 3 without the wealth controls women in joint patrilocal households had 10.5 pp higher chance of giving birth at a health facility compared to their counterpart who lived in nuclear household structure. The same is observed in both NFHS 4 and 5 where we see consistently women in joint patrilocal households have higher chances of giving birth in health facility and the difference goes down once, we control for wealth. 
In case of going for 4+ ANC visits here also we observe a significant difference when it comes to controlling for wealth. We observe in NFHS 3 women in joint patrilocal households had 7 pp more chances of completing 4+ ANC visits compared to their counterparts in nuclear households when controlled for wealth the difference goes down to 0.05 pp. the same is observed for NFHS 4 and 5 where the differences are 7.6 pp before wealth control and 1.2 pp after control and in NFHS 5 the   difference is 5.9 pp before control and 1.6 pp after control. 





Table \ref{tab:sumstats} shows that patrilocal households do better on wealth measures and healthcare measures but worse on autonomy measures. 

\begin{landscape}

\begin{table}[!htbp]
    \centering
    \begin{threeparttable}[t]
        \caption{:  Summary statistics for variables used in mediation analysis}
        \label{tab:sumstats}

        \scriptsize
        \setlength{\tabcolsep}{6pt}
        \renewcommand{\arraystretch}{1}

        \begin{tabular}{lcccccc}
\toprule
 & \multicolumn{2}{c}{NFHS-3} & \multicolumn{2}{c}{NFHS-4} & \multicolumn{2}{c}{NFHS-5} \\\\
\cmidrule(lr){2-3} \cmidrule(lr){4-5} \cmidrule(lr){6-7}
 & Joint & Nuclear & Joint & Nuclear & Joint & Nuclear \\\\
\midrule
\textbf{Currently pregnant women}&&&&&&\\
&&&&&&\\
\textbf{Autonomy measures}&&&&&&\\
\hspace*{2em}Say in own healthcare&0.32&0.51&0.26&0.32&0.18&0.25\\
\hspace*{2em}Say in visits to family/friends&0.32&0.56&0.25&0.35&0.17&0.25\\
&&&&&&\\
\textbf{Wealth measures}&&&&&&\\
\hspace*{2em}Finished floor&0.30&0.40&0.44&0.53&0.52&0.61\\
\hspace*{2em}Electricity&0.50&0.60&0.78&0.87&0.93&0.96\\
\hspace*{2em}Owns radio&0.24&0.34&0.05&0.08&0.04&0.05\\
\hspace*{2em}Owns TV&0.27&0.45&0.46&0.66&0.50&0.70\\
\hspace*{2em}Owns refrigerator&0.06&0.13&0.17&0.32&0.25&0.43\\
\hspace*{2em}Owns bicycle&0.43&0.62&0.46&0.59&0.45&0.55\\
\hspace*{2em}Owns car&0.01&0.03&0.03&0.06&0.04&0.09\\
\hspace*{2em}Uses toilet/latrine&0.30&0.41&0.48&0.58&0.71&0.81\\
\hspace*{2em}Owns land&.&.&0.30&0.28&0.37&0.31\\
&&&&&&\\
\textbf{N}&1,769&2,630&7,747&15,695&6,658&14,226\\
&&&&&&\\
\textbf{Women who gave birth 3--12 months before the survey}&&&&&&\\
&&&&&&\\
\textbf{Healthcare measures}&&&&&&\\
\hspace*{2em}Birth in a health facility&0.34&0.45&0.76&0.85&0.86&0.93\\
\hspace*{2em}4+ antenatal visits&0.31&0.40&0.44&0.54&0.53&0.61\\
&&&&&&\\
\textbf{Wealth measures}&&&&&&\\
\hspace*{2em}Finished floor&0.32&0.42&0.43&0.53&0.48&0.60\\
\hspace*{2em}Electricity&0.53&0.64&0.77&0.87&0.93&0.97\\
\hspace*{2em}Owns radio&0.22&0.35&0.05&0.08&0.03&0.04\\
\hspace*{2em}Owns TV&0.28&0.48&0.46&0.67&0.48&0.70\\
\hspace*{2em}Owns refrigerator&0.06&0.16&0.15&0.31&0.22&0.42\\
\hspace*{2em}Owns bicycle&0.45&0.62&0.45&0.57&0.45&0.55\\
\hspace*{2em}Owns car&0.01&0.03&0.02&0.06&0.04&0.08\\
\hspace*{2em}Uses toilet/latrine&0.34&0.40&0.46&0.57&0.68&0.81\\
\hspace*{2em}Owns land&.&.&0.32&0.30&0.36&0.32\\
&&&&&&\\
\textbf{N}&3,043&4,205&13,477&24,067&11,073&21,707\\
&&&&&&\\
\bottomrule
\end{tabular}


        \begin{tablenotes}
            \item put table notes here
        \end{tablenotes}
    \end{threeparttable}
\end{table}
\end{landscape}


\begin{landscape}

Table \ref{tab:reg_coefs} shows that wealth accounts for the patrilocal advantage in healthcare utilization measures, but not the patrilocal disadvantage in autonomy measures.








\begin{table}[!htbp]
    \centering
    \begin{threeparttable}[t]
        \caption{:  patrilocal coefficient in different regression specifications}
        \label{tab:reg_coefs}

        \scriptsize
        \setlength{\tabcolsep}{6pt}
        \renewcommand{\arraystretch}{2}

        \begin{tabular}{l*{4}{>{\centering\arraybackslash}p{3.2cm}}}
\toprule
 & \multicolumn{2}{c}{sample: currently pregnant women} & \multicolumn{2}{c}{sample: women who gave birth 3+ months before the survey} \\\\
\cmidrule(lr){2-3}\cmidrule(lr){4-5}
 & no say in own healthcare & no say in visits to family/friends & gave birth in a health facility & had 4+ antenatal visits \\\\
\midrule
\textbf{NFHS-3}&&&&\\
\hspace*{2em}no controls&0.184*** \newline (0.013)&0.241*** \newline (0.013)&0.105*** \newline (0.006)&0.070*** \newline (0.006)\\
\hspace*{2em}wealth controls&0.191*** \newline (0.013)&0.242*** \newline (0.013)&0.035*** \newline (0.005)&0.005 \newline (0.005)\\
&&&&\\
\textbf{NFHS-4}&&&&\\
\hspace*{2em}no controls&0.069*** \newline (0.012)&0.101*** \newline (0.012)&0.088*** \newline (0.002)&0.076*** \newline (0.003)\\
\hspace*{2em}wealth controls&0.079*** \newline (0.012)&0.106*** \newline (0.012)&0.045*** \newline (0.002)&0.012*** \newline (0.002)\\
&&&&\\
\textbf{NFHS-5}&&&&\\
\hspace*{2em}no controls&0.061*** \newline (0.013)&0.060*** \newline (0.013)&0.061*** \newline (0.002)&0.059*** \newline (0.003)\\
\hspace*{2em}wealth controls&0.070*** \newline (0.013)&0.069*** \newline (0.013)&0.034*** \newline (0.002)&0.016*** \newline (0.003)\\
\bottomrule
\end{tabular}


        \begin{tablenotes}
            \item put table notes here
        \end{tablenotes}
    \end{threeparttable}
\end{table}
\end{landscape}






\section{Discussion}

\section{Appendix}

\begin{table}[!htbp]
    \centering
    \begin{threeparttable}[t]
        \caption{:  proportion patrilocal by subgroup}
        \label{tab:patrilocal}

        \scriptsize
        \setlength{\tabcolsep}{6pt}
        \renewcommand{\arraystretch}{1.3}

        \input{code/old/tables/table1_patrilocal_by_subgroup}

        \begin{tablenotes}
            \item put table notes here
        \end{tablenotes}
    \end{threeparttable}
\end{table}


\begin{table}[!htbp]
    \centering
    \begin{threeparttable}[t]
        \caption{:  patrilocal coefficient in different regression specifications}
        \label{tab:decomposition}

        \scriptsize
        \setlength{\tabcolsep}{6pt}
        \renewcommand{\arraystretch}{2}

        \begin{tabular}{lccc}
\toprule
Outcome & \makecell{Total change\\(pp)} & \makecell{Share due to change in\\household structure (\%)} & \makecell{Share due to within-\\household changes (\%)} \\
\midrule
No say in own healthcare&-21.59&3.66&96.34\\
No say in visits to family/friends&-24.68&3.96&96.04\\
Four or more antenatal visits&15.79&-5.45&105.45\\
Birth in a health facility&46.74&-2.23&102.23\\
\bottomrule
\end{tabular}


        \begin{tablenotes}
            \item put table notes here
        \end{tablenotes}
    \end{threeparttable}
\end{table}



\bibliographystyle{apalike}
\bibliography{references}

\end{document}